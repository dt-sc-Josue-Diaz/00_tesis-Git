\include{pre/pre-1}

\begin{document}

%\maketitle
\tableofcontents

\addcontentsline{toc}{chapter}{Teoría y definiciones}
\chapter*{Teoría y definiciones}	

%%%%%%%%%%% Introducción
\section*{Introducción}
% Vamos a explicar el panorama de este escrito, de las intenciones del por que de este capítulo. 

A lo largo de este texto utilizaremos los resultados de topología general y teoría de grupos. Nuestra primera meta consta de introducir notación y teoremas correspondientes a espacios topológicos, operadores topológicos, homemorfismos y finalmente llegar a espacios generados. El siguiente paso es usar teoría de grupos y vamos a retomar temas como los homomorfismos de grupos, grupos derivados para terminar en grupos simples. 

Al final de este capítulo, hablaremos brevemente de grupos topológicos, en concreto de un espacio de homeomorfismos de un espacio topológico en donde vamos a demostrar mediante la topología compacto-abierta la continuidad de las operaciones de grupo. La bibliografía en que nos basamos es estándar, de libros de cursos o libros de las editoriales de la UNAM. 

%%%%%%%%%%%%%%%%%%%%%%%%%%%%%%% Espacios topológicos
\addcontentsline{toc}{section}{Espacios topológicos}
\section*{Espacios topológicos}

% Definición de las operaciones conjuntistas y topológica.
Vamos a usar la notación estándar de conjuntos. La pertenencia de un elemento $x$ a un conjunto $X$ se denota por $x \in X$. Dados dos conjuntos $A$ y $B$ denotaremos la contención del conjunto $A$ en el conjunto $B$ por $A \subset B$ y la igualdad de conjuntos $A=B$ representa que se dan las contenciones $A \subset B$ y $B \subset A$.


La unión del conjunto $A$ con $B$ se representa por $A \cup B$, la intersección del conjunto $A$ con el conjunto $B$ se representa por $A \cap B$. $A \setminus B$ representa el conjunto de los elementos que están en $A$ pero que no están en $B$ en particular a $X\setminus B$ le diremos el complemento de $B$ en $X$,  finalmente $\mathcal{P}(X)$ representa el conjunto potencia de $X$, es decir, la familia $\mathcal{P}(X)=\{A: A \subset X\}$.  Sean $X$, $Y$ conjuntos y $f:X \to Y$ una función, para cualquier $A \subset X$ el conjunto,
 
 \begin{align*}
 % Iamgen directa
 f(A)=\{y \in Y : \text{ existe }x \in A \text{ y } f(x)=y\}
 \end{align*}

será llamado imagen directa de $A$ bajo $f$. Por otro lado  dado $B \subset Y$ el conjunto 

\begin{align*}
% Imagen inversa
f^{-1}(B)=\{x \in X :  f(x) \in B \}
\end{align*}

será llamado la imagen inversa del conjunto $B$ bajo $f$. 

\begin{ob}
% nota sobre tomar imagenes inversas o directas
En general tomar la imagen inversa de una imagen directa o en orden alterno, no se obtiene como resultado el mismo conjunto. Pero se tiene que

	
	\begin{align*}
	f(f^{-1}(B)) \subset B \\
	A \subset f^{-1}(f(A)) .
	\end{align*}

La primera contención es igualdad si $f$ es sobreyectiva y la segunda contención es igualdad si $f$ es inyectiva, es claro que si $f$ es biyectiva tenemos las dos igualdades. 
\end{ob}

\begin{ej}
% Ejemplo concreto, quizas se vaya a eliminar. 
Sea $f:\{1,2,3,4\} \to \{1,2,3,4\}$ dada por, $f(1)=f(3)=1$, $f(2)=f(4)=3$. Esta función no es inyectiva ni sobreyectiva. Consideremos a $\{1,4\}$, notemos que $f^{-1}(\{1,4\})=\{1,3\}$ pero $f(f^{-1}(\{1,4\}))=\{1\}$. Por otro lado, consideremos a $\{1,2\}$ y notemos que $f(\{1,2\})=\{1,3\}$  pero que $f^{-1}(\{1,3\})=\{1,2,3,4\}$.
\end{ej}

En la práctica es usual no indicarse el dominio e imagen de una función pues implícitamente se da a entender que el lector no tiene inconveniente.

\begin{df}
% Definición de topología. 
Sea $X$ un conjunto y $\tau \subset \mathcal{P}(X)$ una familia de subconjuntos de $X$. Decimos que $\tau$ es una \textbf{topología} para $X$ si $\tau$ cumple que;

	\begin{enumerate}
	\item $\emptyset$, $X$ son elementos de  $\tau.$
	
	\item Para cada subfamilia finita de $\tau$, $\{A_i\}_{i=1}^n$ se tiene que $\bigcap_{i=1}^n A_i$ es un elemento de $\tau.$ 
	
	\item Para cada subfamilia $\{A_i\}_{i \in J}$ donde $I$ es un familia de índices arbitrario, se tiene que $\bigcup_{i \in I} A_i$ es un elemento de $\tau$.
	\end{enumerate}

Por \textbf{espacio topológico} nos referimos a un par $(X,\tau)$ donde $X$ es un conjunto y $\tau$ es una topología para $X$. Denotaremos al espacio $(X, \tau)$ por $X_{\tau}$. A los elementos $U$ de $\tau$  les llamaremos \textbf{conjuntos abiertos}. Un conjunto $V$ se dice \textbf{cerrado} si es complemento de un conjunto abierto es decir existe $U$ conjunto abierto tal que $V=X \setminus U$.  
% Final de la definición
\end{df}


\begin{nt}
% Nota sobre como vamos a llamar a la numerabilidad 
La segunda condición se conoce como \textbf{cerradura bajo intersecciones finitas} o que \textbf{familia es cerrada bajo intersecciones finitas.} La tercera condición se conoce como \textbf{cerradura bajo uniones arbitrarias} o que \textbf{la familia es cerrada bajo uniones arbitrarias.}
% Fin de nota
\end{nt}


\begin{ej}
% Espacios discreto e indiscreto
Sean $X$ un conjunto y la familia $2^X$, el par formado por $(X, 2^X)$ es un espacio topológico y es llamado \textbf{espacio discreto}. Por otro lado sea $\tau=\{\emptyset, X\}$, esta familia cumple la definición de topología, el espacio $(X, \tau)$ es llamado \textbf{espacio indiscreto}.
%fin ejemplo
\end{ej}

Además, si $X$ contiene más de un punto, las familias $2^X$ y $\tau$ del ejemplo anterior son distintas pero se da la contención  $\tau \subset 2^X$ notemos que en consecuencia un conjunto puede tener mas de una topología y entre ellas distintas. 

La topología guarda información importante del conjunto $X$ que nos puede ayudar a distinguir propiedades. En el ejemplo, los espacios discreto e indiscreto no distinguen mucho sobre $X$, estas dos topologías en este aspecto son triviales.

\begin{cn}
% Presindir de la notación de topología cuando no sea necesario
Cuando el contexto sea claro sobre el espacio topológico vamos prescindir de la notación de la topología y simplemente diremos que $X$ es espacio topológico. 
%Fin de convencion
\end{cn}

Ahora, un resultado que nos permitirá hablar de espacios topológicos en en subconjuntos, esto nos permitirá de hablar sin ambigüedades respecto a los ejemplos concretos que estudiaremos. 

\begin{te}
% Topología del subespacio
Sea $X$ un espacio topológico $Y \subset X$ entonces $\tau_Y =\{A \cap U: U \in \tau\}$ es una topología para $Y$. Por \textbf{subespacio} $Y$ de $X$ nos referimos al espacio $(Y, \tau_Y)$.
% Fin de teorema
\end{te}

\begin{cn}
% Adevertencia de no detallar las bases y los sistemas de vencindades
El tema que no vamos a estudiar los temas de bases y sistema de vecindades. Para estos temas tenemos en la bibliografía \cite{top_prieto} capitulo 2, página 58 para bases de vecindades y capitulo 3, página 73 para bases para una topología.
\end{cn}

La familia de conjuntos que a continuación vamos a detallar, nos da una manera de obtener espacios topológicos mediante un sistema de vecindades. Este ejemplo es muy rico en propiedades y en ocasiones un fuerte contraejemplo. 


\begin{ej}
% Inicio del ejemplo de  Moore
Sea $X= \{(x,y) \in \mathbb{R}^2: y \geq 0 \}$. Definimos una familia de conjuntos mediante las siguientes condiciones; 

\begin{enumerate}
	\item dado $(x,y) \in X$ y $r>0$ si $y > 0$ definimos el conjunto,

	\begin{align*}
	B_r((x,y),r)=\{(w,z) \in X : \| (x,y)-(w,z) \| < r \},	
	\end{align*}

donde hacemos énfasis que es la norma usual de $\mathbb{R}^2$ y $z \neq 0$. 

	\item Si $y=0$ tomamos el conjunto ;

	\begin{align*}
	\beta((x,0))=B_r((x,r),r) \cup \{(x,0)\}.
	\end{align*}
\end{enumerate}

Hemos definido una familia de conjuntos sobre los puntos de $X$, esta familia de conjuntos induce una base para una topología sobre $X$ y este espacio es conocido como \textbf{plano de Moore}.

Notemos que en este espacio $\mathbb{R} \times \{0\} \subset X$, pero la restricción al subespacio $\mathbb{R} \times \{0\}$ nos da un conjunto discreto mientras que la restricción al plano $\{(x,y)\in X :y > 0 \}$ es el plano positivo en $\mathbb{R}^2.$
% Final del ejemplo de Moore.
\end{ej}

Vamos a referir el tema de axiomas de separación a los libros de \cite{top_prieto} en el capítulo IX y \cite{top_willd} capítulo 5 puesto que hay que en los axiomas de separación hay que ser claros con la nomenclatura $T_i$. Pero nosotros vamos a trabajar con espacios métricos. Por lo que basta que hablemos de los espacios Hausdorff. 





Para operadores topológicos utilizaremos la siguiente notación. 

\begin{df}
% Vecindades de puntos y Operadores; intererior, clausura y frontera.
Sea $X$ un espacio topológico y $U$ subconjunto de $X$.

\begin{itemize}
	% Vecindad
	\item Diremos que $U$ es \textbf{vecindad abierta} de un punto $x$ a, si $x \in U$ y $U \in \tau$. A la familia de vecindades de un punto $x$ la denotaremos mediante $\mathcal{N}_x$.
 	% Interior
	\item Al conjunto \textbf{interior} de $A$ en $X$ lo denotaremos por 
 
  $$\Int_X(A)=\{x \in A: \text{ existe } U \in \mathcal{N}_x \text{ tal que } U \subset A\}.$$
  
	% Clausura
	\item  Al conjunto  \textbf{clausura} de $A$ en $X$ lo denotaremos por, 
  
$$\Cla_X(A)=\{x \in A: \text{para toda } U \in \mathcal{N}_x \text{ se cumple  que } U \cap A \neq \emptyset\}.$$
	
	% Frontera 
	\item Denotaremos por  $\Fro_X(A)$ al conjunto $\Cla_X(A^c) \cap \Cla_X(A) $ a este conjunto le llamaremos la 
  \textbf{frontera} de $A$ en $X$.
 \end{itemize}
% Fin definición
\end{df}


\begin{cn}
% Los operadores presinden de la topología. 
Cuando el contexto lo permita simplemente denotaremos por $\Int(A)$ al interior, $\mathrm{CL}(A)$ la clausura y  $\Fro(A)$ a la frontera de $A$ en $X$.
%Fin del convenio de presindir de la topología.
\end{cn}

\begin{df}
Sea $X$ un espacio topológico. Decimos que $X$ es un espacio $T_2$ o Hausdorff si para cada $x$ e $y \in X$ distintos existen $U$ vecindad de $X$ Y $V$ vecindad de $y$ tal que $U \cap V = \emptyset.$
\end{df}

En esta parte de la topología nos interesa estudiar los espacios mediante de las propiedades de $X$ o de $\tau$ (incluso ambas) y compararlas mediante las de otros espacios, esto es, poder demostrar que un espacio desconocido tenga propiedades de espacios ya conocidos, por lo que podemos clasificar por medio relaciones entre espacios y trabajar con un espacio ya conocido o mas estudiado. Esta noción coloquialmente conocida como trabajar salvo homeomorfismo. 

\begin{df}
% Continuidad, Homeomorfismos
Sea $X$ espacio topológico y una función $h:X \to X$. 
\begin{enumerate}

	\item Dado $B$ subconjunto de $X$, $h|_B$ denotará la restricción $h:B \to h(B)$ de $h$ a $B$. 
	
	\item  Decimos que $h$ es \textbf{continua} si para cada conjunto abierto $U$ se cumple que $h^{-1}(U)$ es un conjunto abierto.
	
	\item  Sea $h$ una función continua y biyectiva. Decimos que $h$ es un \textbf{homeomorfismo} si la función inversa de $h$, $h^{-1}:X \to X$ es continua. 	
\end{enumerate}
% Fin definición Continuidad, Homeomorfismos
\end{df}


\begin{ej}
Sean $X$ un conjunto con mas de un punto y las topologías $2^X$ y $\tau=\{\emptyset, X\}$. Consideremos la función $$Id_X:X_{\tau} \to X_{2^X}$$ dada por $$Id_X(x)=x.$$
Notemos que para cada $x \in X$ el conjunto $\{x\}$ es un conjunto abierto en $X_{2^X}$, pero $Id_X^{-1}(\{x\})= \{x\}$ no lo es en $X_\tau$. Sin embargo reescribiendo  la función anterior de la siguiente manera,  $$Id_X:X_{2^X} \to X_{\tau}$$ dada por $$Id_X(x)=x,$$
si es continua pues $Id_X^{-1}(X)=X$ el cual es un conjunto abierto, el caso $\emptyset$ es trivial. En particular, una función continua y biyectiva no siempre tiene inversa continua. 
\end{ej}

El siguiente resultado es conocido como el \textbf{lema de pegadura}

\begin{pr}
Sean $\mathcal{U}= \{A_\alpha :\alpha \in A\}$ una familia de abiertos tales que $\bigcup \mathcal{U} = X$ y $f:X \to Y$ una función. Si $f_\alpha=f|_{A_\alpha}$ es continua para todo $\alpha \in A$ entonces $f$ es continua.  
\end{pr}


Una curva cerrada simple es una trayectoria $\alpha$ tal que su restricción al intervalo abierto $(0,1)$ es inyectiva pero $\alpha(0)= \alpha(1)$. 

\subsection*{Invariantes topológicos}

Los invariantes topológicos son propiedades que una topología tiene y que éstas se preservan mediante homeomorfismos. Esto es sean $X$, $Y$ espacios y $\mathcal{P}$ una propiedad topológica tal que $X$ posee a $\mathcal{P}$ se dice que $\mathcal{P}$ es un \textbf{invariante topológico} si para todo homemorfismo $h:X \to Y$,  $Y$ tiene la propiedad $\mathcal{P}$.


Además es importante mencionar el hecho de que un espacio pueda tener una propiedad $\mathcal{P}$ para una topología en particular, y no poseer la propiedad con otra topología definida en el mismo conjunto. Veremos dos de las  propiedades más estudiadas e 
importantes del análisis moderno. 

\subsection*{Compacidad}

\begin{df}
Sean $X$ un espacio topológico y  $A$ subconjunto de $X$. Una familia de abiertos $(U_i)_{i \in I}$ es una \textbf{cubierta abierta} para $A$ si $A \subset \bigcup_i U_i.$ Por \textbf{subcubierta abierta} nos referimos a una subfamilia $(U_{i_j} )_{j \in J}$ de $(U_i)_{i \in I}$ tal que $ A \subset \bigcup_j U_{i_j}.$

Decimos que un espacio topológico es \textbf{compacto} si para toda cubierta abierta de $X$ existe una subcubierta abierta y finita para $X$. Decimos que un subconjunto $A \subset X$ es compacto si lo es como subespacio.
\end{df}

Un teorema importante respecto a funciones continuas y conjuntos compactos es el siguiente teorema y puede consultarse en la  página 121 de \cite{top_willd}.
 
\begin{te}
Sea $C$ un subconjunto compacto de $X$, para toda función continua $f:X \to Y$ el conjunto $f(C)$ es un conjunto compacto en $Y$.
\end{te}


Esto puede encontrarse en página 120 de \cite{top_willd}. 

\begin{te} 
Sean $X$ un espacio y un subconjunto $B \subset X$. 

	\begin{enumerate}
	\item Si $X$ es compacto y si $B$ es cerrado entonces es compacto. 
	\item Si $X$ es $T_2$ y si $B$ es compacto entonces $B$ es cerrado. 
	\end{enumerate}
	
\end{te}

Una demostración puede encontrarse en \cite{top_prieto}, capítulo VIII, teorema 1.22. En palabras simples, la imagen continua de un conjunto compacto es compacta. Para el tema de funciones continuas. Ahora un resultado en $\mathbb{R}^n$ sobre la continuidad y espacios compactos.

\begin{te}
 Sean $f:A \subset \mathbb{R}^n \to \mathbb{R}^m$ y $K \subset A $. Si $f$ es una función continua y $K$ es un conjunto compacto entonces $f$ es uniformemente continua sobre $K$.
\end{te}

 Una demostración de este teorema se puede encontrar en \cite{cal_Paez} capitulo 2, Teorema 2.5

%%% conexidad
\subsection*{Conexidad}
 Decimos que $X$ es \textbf{disconexo} si existe abiertos $U$ y $V$ ajenos y no vacíos tales que $X = U \cup V$. Decimos que un espacio $X$ es \textbf{conexo} si no es disconexo. Un subconjunto $Y \subset X$ es conexo (o disconexo) si lo es como subespacio.

El siguiente teorema se encuentra en \cite{top_willd} como Theorem 26.3.

\begin{te}
Sean $X$, $Y$ espacios topológicos y $f:X \to Y$ una función continua. Dado $C$ un subconjunto conexo de $X$ se tiene que $f(C)$ es un conjunto conexo.
\end{te}

Lo anterior se resume en que la imagen continua de un conjunto conexo es conexa. Vamos a continuar con el tema de la topología métrica, es un tema que con el cual se introducen los cursos de análisis matemático y nuestras primera nociones de topología. 

\addcontentsline{toc}{section}{Topología métrica}
\section*{Topología métrica}
Un \textbf{espacio métrico} es un par $(X, d)$ donde $X$ es un conjunto y $d:X \times X \to [0, \infty)$ es una función que satisface las siguientes condiciones;

\begin{enumerate}
\item $d(x,y)=0$ si y sólo si $x=y$,
\item $d(x,y)=d(y,x)$
\item $d(x,y) \leq d(x,z)+d(y,z)$
\end{enumerate}


A la función $d$ le llamamos una \textbf{métrica} para $X$. Al par $(X,d)$ le denotaremos por $X_d$. 

\begin{ob}
La notación $X_d$ hacemos énfasis es que $X$ es el conjunto con métrica $d$, más adelante veremos que un espacio métrico es un espacio topológico por lo que $X_d$ es el equivalente a $X_\tau$ donde $\tau$ es la generada por la métrica salvo que ahora indicamos la cualidad métrica de esta topología.  
\end{ob}

\begin{ej}
% Ejemplo de que Rn es un espacio métrico.
 Dados $\vec{a}$, $\vec{b} \in \mathbb{R}^{n}$, la norma euclidiana,
$$d(\vec{a},\vec{b}):=\| \vec{a}-\vec{b} \|$$
define una métrica en $\mathbb{R}^{n}$.
\end{ej}

En general espacios normados son espacios métricos. 
 
\begin{ej}
% Toda métrica es equivalente a una métrica acotada.
Sean $X$ un conjunto no vacío y $d : X \times X \to\mathbb{R}$ definida mediante:
$$d(a,b)=\left\{
\begin{array}{lcc}
1, & \text{si} & a \neq b; \\
0, & \text{si} & a=b. \\
\end{array}
\right.$$

$d$ es conocida como la \textbf{métrica discreta} y  $X_{d}$ como \textbf{espacio métrico discreto}.
\end{ej}


\begin{ej}\label{ejem:metrica-acotada}
Sea $X_{d}$ espacio métrico. Definimos,
$$\bar{d}(a,b)=\left\{
\begin{array}{lcc}
1 & \text{ si } & 1 < d(a,b);   \\
d(a,b) & \text{ si } & d(a,b)<1. \\
\end{array}
\right.
$$
$\bar{d}$ es una métrica para $X$ con $\bar{d}(a,b)\leq 1$ para cuales quiera $a$, $b \in X.$ 
\end{ej}


En general tenemos que una métrica es equivalente a una métrica acotada. Esto es un paso interesante, puesto que en $\mathbb{R}^n$ un espacio es compacto si y solo si es cerrado y acotado, el conocido teorema de Heine-Borel. En espacios métricos podemos construir un espacio discreto, cerrado y acotado de tal manera que no sea compacto. 


\label{def:top-espacios-metricos}
\begin{df}
% Definición de bolas.
Sean $\varepsilon > 0$ y $x\in X_d$. Definimos a la \textbf{bola abierta} de centro $x$ y radio $\varepsilon$  como

$$B_{d}(x,\varepsilon)=\{ y \in X: d(x,y)< \varepsilon \}.$$

De manera parecida definimos a la \textbf{bola cerrada} de centro $x$ y radio $\varepsilon$  como
$$\overline{B}_{d}(x,\varepsilon)=\left\lbrace y \in X: d(x,y)\leq \varepsilon \right\rbrace.$$
\end{df}

Para diferenciar las métricas que $X$ pueda tener (y por tanto las bolas de $X$) denotaremos $B_{d}(x,\varepsilon)$ y diremos que es una d-\textbf{bola.} En caso de que no haya confusión simplemente denotaremos $B(x,\varepsilon)$. Tenemos los conceptos necesarios para definir una topología para un espacio métrico.

\begin{pr}\label{prp:bolas-bse}
% Las bolas inducen una base para una topología.
Sea $X$ espacio métrico. La familia $\lbrace B_{d}(x,\varepsilon): x \in X, \varepsilon > 0\rbrace$ de todas las d-bolas es base para una topología sobre $X$.
% Fin de topología métrica.
\end{pr}
Esto puede demostrarse usando el teorema 5.4 de \cite{top_willd}.


\begin{te}
Si $x$, $y \in X_{d}$ con $ x\neq y $ entonces existen vecindades $U$ de $x$ y $V$ de $y$ tales que $U \cap V=\emptyset$.
\end{te}


Dados $x$, $y$ puntos distintos en un espacio métrico. Consideremos a $D=d(x,y)$ del hecho que $B(x,D/2)\cap B(y,D/2)=\emptyset$.


\begin{ob}\label{obs:espacios-metricos-son-hausdorff}
El teorema anterior nos dice que los espacios métricos son espacios Hausdorff.
\end{ob}

Terminamos esta parte con la siguiente definición que es muy usada en los artículos de \cite{kras}, \cite{why} y \cite{moore}.

\begin{df}
Decimos que $X$ es un \textbf{continuo} si es un espacio métrico, compacto y conexo.
\end{df}

Al estudiar espacios métricos nos vemos en la opción de no estudiar los axiomas de separación, que es una parte muy importante de la topología general. De nuevo, recomendamos el siguiente texto donde se puede profundizar este tema. Es claro que no hablar de este tema nos deja un estudio sesgado, pero usaremos solo un axioma de separación hasta la parte de la topología del grupo de homeomorfismos de la circunferencia. Incluso, solo es necesario concordar el tercer axioma separación. 

\addcontentsline{toc}{section}{Topología cociente}
\section*{Topología cociente}

En topología se estudia distintas maneras de obtener espacios topológicos como el subespacio, los productos pero otra mas compleja es la topología generada por funciones o familia de funciones, en sí es obtener un espacio topológico donde una función o una familia de funciones son continuas. En \cite{top_salicrup}, se explica de manera completa este tema, desde el capitulo 3 hasta el 5. Sin embargo tomaremos las definiciones de \cite{top_prieto} y de \cite{top_juan} pues son mas extensos en detalles y de lectura mas cómoda. 

\begin{df}
Sean $X$ un espacio topológico, $Y$ un conjunto y $f:X \to Y$ una función sobreyectiva. La topología mas fina en $Y$ para la cual $f$ es continua se llama \textbf{topología de identificación} inducida por $f$. A $f$ se le llama \textbf{identificación}.
\end{df}

Dado $X$ un espacio y $A \subset X$, consideremos a $Y=(X\setminus A) \cup \{A\}$, el subespacio $A$ ha sido colapsado a un punto $\{A\}$, sea

\begin{align*}
q(x)= \begin{cases}
x \text{ si } x \in X\setminus A \\
A \text{ si } x \in A.
\end{cases}
\end{align*}

la topología final inducida por $q$ es llamada la \textbf{topología identificación}; el espacio resultante se llama \textbf{espacio de identificación} y a $q$ se le llama \textbf{identificación}. 



Los resultados sobre la topología de identificación se pueden consultar en \cite{top_prieto} página 87. Veamos ahora los espacios cocientes. Sean $X$ espacio topológico y $\sim$ una relación de equivalencia en $X$. Recordemos que una relación de equivalencia induce una partición de un conjunto en subconjuntos ajenos que son las clases de equivalencia, las cuales se definen por $[x]=\{y \in X : x \sim y\}$. La familia de las clases de equivalencia es denotado por $X/ \sim$ es decir, $$X/ \sim=\{[x]: x \in X\},$$  

tenemos una función sobreyectiva,

\begin{align*}
f:X \to X / \sim
\end{align*}

dada por

\begin{align*}
f(x)=[x].
\end{align*}

\begin{df}
Con la topología de identificación inducida por $f$ a la familia $X/ \sim$ se llama espacio cociente de $X$ bajo la relación $\sim$. 
\end{df}

\begin{ej}
Sea $f: \mathbb{R} \to S^1$ la función exponencial, $f(s)=e^{2\pi i s}$ es una identificación talque $f(s)=f(t)$ si y solo si $t-s \in \mathbb{Z}$.
\end{ej}

El siguiente teorema se encuentra en \cite{top_prieto} como teorema IV.2.16.

\begin{te} 
Sea $f:X \to Y$ una identificación. Si se define una relación tal que $x \sim y$ si y solo si $f(x)=f(y),$ entonces $f$ determina un homeomorfismo $\tilde{f}:X/ \sim \to Y.$
\end{te}


\begin{ej}\label{ej:Cir_un_coc}
Consideremos a $\mathbb{Z}$ como subespacio de $\mathbb{R}$ y la relación; $x \sim y$ si $x-y \in \mathbb{Z}$. Esta relación es de equivalencia. En efecto, 

\begin{enumerate}
	\item $x-x=0 $ para todo $x \in \mathbb{R}$.
	\item Supongamos que $x-y \in \mathbb{Z}$, entonces claramente $y-x=-(x-y) \in \mathbb{Z}$ y por tanto $y \sim x$.
	\item Si $x-y \in \mathbb{Z}$ y $y-z \in \mathbb{Z}$ resta notar que $$x-z=x-y+y-z$$
el cual es un numero entero. Por tanto $x \sim z$.
	\end{enumerate}


Del teorema anterior y de la función exponencial se tiene que $(\mathbb{R}/\mathbb{Z}) \cong S^1.$ La función  $\pi: \mathbb{R} \to \mathbb{R}/ \sim$ dada por $\pi (x)=[x]$ está bien definida y es una función cociente. Definimos a $\Hom(\mathbb{R})$ los homeomorfismos monótonos crecientes de la recta que cumplen la propiedad, 

$$\tilde{f}(x+1)= \tilde{f}(x)+1.$$

Este homeomorfismo induce un homeomorfismo,

\begin{align*}
f:\mathbb{R}/ \mathbb{Z} \to  \mathbb{R}/ \mathbb{Z}
\end{align*}
 
dado por $f=\pi(\tilde{f}).$ Notemos que de esto se induce un homeomorfismo de la circunferencia en la circunferencia $f:S^1 \to S^1$. Un homeomorfismo de este tipo se dice que \textbf{preserva orientación} si es creciente y no la preserva si es decreciente. 
\end{ej}

Concluimos esta sección con toda la notación referente a topología que vamos a usar o se usan en los artículos consultados. En caso dado se use otro resultado será tratado en la respectiva sección. 

%%%				Grupos				%%%%% 

\addcontentsline{toc}{section}{Grupos}
\section*{Grupos}
% Introducir lo importante que es la teoria de grupos para la exposición del tema. 
En este texto estudiaremos resultados de teoría de grupos, para ello planteamos la notación necesaria para desarrollar dichos temas. Se recomienda al lector para el desarrollo de este tema las siguientes referencias bibliográficas, para la secuencia del desarrollo de los temas usamos a \cite{alg_grove} y la notación de \cite{alg_ii}.

\begin{df}
% Definición de grupo
Sea $G$ un conjunto no vacío. Por grupo nos referimos a una terna $(G, \circ, e)$ donde $\circ:G \times G \to G$ una operación en $G$ y $e$ un elemento distinguido de $G$ tales que;

\begin{enumerate}
	\item La operación $\circ$ es asociativa.
	
	\item para cada $g \in G$ existe $h \in G$ tal que $ \circ (g,h)= \circ(h ,g)=e$. A $h$ se le llamara un inverso de $g$. 
	
	\item Para todo $g \in G$ se cumple que $\circ(g ,e) =  \circ(e,g) = g.$
\end{enumerate}
% Fin, definición de grupo
\end{df}

\begin{cn}
% Evitar lo básico de grupos
Para evitar entrar en resultados introductorios de grupos como con los teoremas de la unicidad de los elementos $e$ y los inversos recurrimos a \cite{alg_grove} y además  denotaremos simplemente por $gh$ a la operación $\circ(g,h)$, en otras palabras, hacemos el abuso de notación,  $\circ(g,h):=gh.$
%Fin de evitar lo básico de grupos
\end{cn}

Una cuestión muy sencilla es que si tomamos un conjunto con un solo elemento $G=\{g\}$ de manera simple podemos definir, $\circ(g,g)=g$ por lo que es $g$ es el elemento identidad y $g$ es su propio inverso bajo $\circ$ y $\{g\}$ es un grupo. 

Pero, si $G$ es un grupo con mas de un elemento, no es necesario que bajo la operación de $G$ se tenga que $gg=g$. Aunque si pasa cuando $g=e$, por lo que tenemos una clase de no vacía de grupos contenidos en un grupo. Lo anterior muestra la necesidad de los inversos y del elemento $e$ en los posibles grupos contenidos (como conjuntos) en un grupo arbitrario $G$.

\begin{df}
% Definición de subgrupo.
Sea $G$ un grupo y subconjunto $H$ de $G$. Decimos que $H$ es \textbf{subgrupo} de $G$ denotado por $H \leq G$, si la operación $\circ:H \times H \to G$ es una operación de grupo en $H$.	
% Fin de definición de subgrupo. 
\end{df}

En otras palabras, $H$ es por si mismo un grupo. Sea $e_H$ la identidad bajo $H$ entonces para $e_G$ la identidad de $G$ se tiene por definición que $e_H e_G=e_H$ en $G$ y también en $H$, por la unicidad de la identidad, $e_G$ es la identidad en $H$ y ahora sin ambigüedad, $e_H=e_G$ para todo subgrupo de $G$, por simplicidad diremos a $e$ para referirnos a $e_G$. Sea $h \in H$ y $h_H^{-1}$ su inverso en $H$, notemos que $h_G^{-1}h=e=h_H^{-1}h$. Finalmente note que $\{e\}$ es un subgrupo contenido en todo subgrupo de $G$. 
		
\begin{ob}
% La intersección de grupos es grupo. Grupo generado por un conjunto.
Sea $X \subset G$ denotemos por $\langle X \rangle$ a la intersección de los subgrupos de $G$ que contienen a $X$. $\langle X \rangle$ es un subgrupo de $G$ que contiene a $X$. Dicho grupo es llamado el \textbf{grupo generado por} $X$. Si $X$ es finito, $X=\{ a_1, \cdots , a_n\}$ entonces $\langle X \rangle$ es denotado por $\langle a_1, \cdots, a_n \rangle$. Más aún si $X=\emptyset$ notemos que $\langle X \rangle=\{e_G\}$ el grupo trivial.
\end{ob}

El siguiente resultado es tomado de \cite{alg_i} como proposición 3.3.1.

\begin{pr}
% Propiedad mininal del grupo generado.
Si $H$ es un subgrupo que contiene a $X$ entonces $\langle X \rangle \subset H$.
% Fin de la propiedad minimal.
\end{pr}

La proposición anterior nos da una propiedad minimal respecto a la contención de $X$. En otras palabras $\langle X \rangle$ es el mínimo subgrupo que contiene a $X$.

\begin{df}
% Definición de palabra.
Sea $G$ un  grupo y $X \subset G$ no vacío. Una \textbf{palabra} en $X$ es un elemento $w \in G$ de la forma
\begin{align*}
w=x_1^{a_i} \cdots x_n^{a_n},
\end{align*}
donde $n \in \mathbb{Z}^+$, $x_i \in X$ y $a_i \in \mathbb{Z}$ para toda i. Denotamos por $\mathbf{W}(X)$ el conjunto de todas la palabras de  $X$. Observemos que $X \subset \mathbf{W}(X)$ además nos es conveniente definir a $\mathbf{W}(\emptyset) :=\{e_G\}$.
% Fin de la definición de palabra.
\end{df}

El siguiente resultado puede encontrarse en imate página 85 3.3.4

\begin{te}
% El grupo generado coincide con el grupo de palabras de un conjunto.
Sea $X \subset G$ se tiene que $\langle X \rangle= \mathbf{W}(X)$.
% Fin de teorema
\end{te}

De esta manera tenemos una descripción de los elementos de $\langle X \rangle$ es decir, para cada $w \in \langle X \rangle$ se tiene una representación mediante, 

\begin{align*}
w=x_1^{a_i} \cdots x_n^{a_n},
\end{align*}

incluso se puede tomar a los elementos $a_i \in \{-1,1\}$ en algunos caso es importante tener la simetría de $X$, el conjunto $X^-=\{x^{-1}| x \in X \}$ pero dependerá de la utilidad de la descripción. Ahora estableceremos la notación para clases laterales que nos permitirá desarrollar el resto de nuestro trabajo. 	
	
\begin{df}
% Definición de clases laterales.
Sea $G$ un grupo y subconjuntos $H$, $K$ de $G$.

\begin{enumerate}
		
	\item El conjunto $KH$ se define como  $KH=\{kh: k \in K \text{ y } h \in H \}$. En particular cuando $K$ conste de un solo elemento, $K=\{k\}$ denotaremos al conjunto $KH$ por $kH$.	

	\item Para todo $g \in G$ el conjunto $gH=\{gh:h \in H \}$ le llamaremos \textbf{clase lateral izquierda}; análogamente por \textbf{clase lateral derecha} nos referimos al conjunto $Hg=\{hg:h \in H\}$ . A la familia  $H_{der} \setminus G=\{gH:g \in G\}$
le llamaremos la familia de clases laterales derechas de $H$ y análogamente a $H_{izq}\setminus G=\{Hg:g \in G\}$
le llamaremos la familia de clases laterales derechas de $H$. 
\end{enumerate} 
% Fin definición de clases laterales
\end{df}	

\begin{ob}
% Las clases laterales no coinciden. 
De la definición anterior tenemos unas observaciones. 

\begin{enumerate}
	\item Las clases de un conjunto forma una partición de $G$ y por tanto las clases inducen una relación de equivalencia.
	
	\item Más aún existe una correspondencia entre clases laterales. Sea 
	
 	\begin{align*}
	\varphi:G/H_d & \to G/H_i \\
	gH & \mapsto Hg^{-1}
	\end{align*} 
 
esta es una función cambia una clase lateral derecha a una izquierda. Veamos que es una función biyectiva. La función es sobre, pues para $Hg$ y para el elemento $g \in G$ tenemos que $g^{-1}$ es tal que $\varphi(g^{-1}H)=Hg$. Resta ver que es inyectiva, sean $g_1H$ y $g_1H$ clases derechas y supongamos que
	
	 $$Hg_1^{-1}=\varphi(g_1H)=\varphi(g_2H)=Hg_2^{-1}$$ 
	
	Sea $g_1h_1 \in g_1H$ notemos que

	\begin{align*}
	g_1h_1(h_1^{-1}g_1^{-1})=e	
	\end{align*}
	
	como $Hg_2^{-1}=Hg_1^{-1}$ existe $h_2$ tal que $h_2g_2^{-1}=h_1^{-1}g_1^{-1}$, así en la ecuación previa tenemos que, 
	
	\begin{align*}
	g_1h_1(h_2g_2^{-1})=e	
	\end{align*} 
	es decir, $g_1h_1=g_2h_2^{-1}$ dando la contención $g_1H \subset g_2H$, la contención recíproca es idéntica se obtiene que $g_1H=g_2H$. Concluimos que $\varphi$ es biyectiva. 
	
	Sin embargo, esta correspondencia no deja en claro que $gH=Hg$ un ejemplo de esto se puede encontrar en 111 imate.
	
	 \item Dado $g \in G$ y $h \in H$ si $ghg^{-1} \in H$ entonces existe $h_1 \in H$ tal que $ghg^{-1}=h_1$ es decir, $gh=h_1g$  que de acuerdo con la notación de clases, cada elemento de una clase izquierda es un elemento de la correspondiente clase derecha, de manera precisa $gH=Hg$.
	 \end{enumerate}
	\end{ob}
	
La situación anterior es sorprendente y no solo es una curiosidad matemática, esto nos da la estructura de como clasificar grupos. 
	
\begin{df}
% Conjugados, grupo normal. 
Sea $G$ un grupo y $H$ subgrupo de $G$.
	
\begin{enumerate}
    \item Dado $g \in G$, denotaremos a $gHg^{-1}$ por $H^g$, en particular cuando $H=\{h\}$ a $gHg^{-1}$ lo denotaremos simplemente por  $h^g$ al elemento $h^{g}$ le llamaremos el \textbf{conjugado} de $h$ por $g$. 
	\item  Decimos que $g$ \textbf{normaliza} a $H$ si $H^{g} \subset H.$ Al conjunto $N_G(H)=\{g \in G: gH=Hg\}$ le llamamos el \textbf{normalizador} de $H$ en $G$.
	\item Si $G=N_G(H)$ decimos que $H$ es \textbf{normal} en $G$ y denotaremos esto por $H \unlhd G.$
		
\end{enumerate}
% Fin de conjugados, grupo normal.
\end{df}
	
\begin{ob}
% Observaciones sobre tabjar el conjugado como exponente.
Sean $g_1,$ $g_2 $ y $H$ conjunto. Notemos que

\begin{align*}
H^{g_1g_2} &=g_1g_2H(g_1g_2)^{-1}=g_1g_2Hg_2^{-1}g_1^{-1}\\
&=(g_2Hg_2^{-1})^{g_1} \\
& = (H^{g_2})^{g_1}
\end{align*}

Donde $H^{g_1g_2}$ no necesariamente es $(H^{g_1})^{g_2}$.
\end{ob}


Los homomorfismos de grupos son funciones importantes, estas funciones permiten preservar la estructura de grupo a otro justo como los homeomorfismos entre espacios topológicos.
	
\begin{df}
% Definición de homomorfismo de grupos y biyecciones
Sean $(G,\circ_G,e_G)$ y $(H,\circ_H,e_H)$ dos grupos. 

\begin{enumerate}
	\item Una función $\varphi:G \to H$ es un \textbf{homomorfismo} de grupos si $$\varphi(\circ_G(g_1,g_2))=\circ_H(\varphi(g), \varphi(h)).$$
	
	\item El \textbf{núcleo} de un homomorfismo $\varphi$ es el  conjunto $$\mathrm{Null}(\varphi)=\{ g \in G: \varphi(g)=e\}.$$
	
	\item Diremos que un homomorfismo de grupos $\varphi$ es:
	\begin{itemize}
    		\item \textbf{Monomorfismo:} si $\varphi$ es inyectiva.
    		\item \textbf{Epimorfismo:} si $\varphi$ es sobreyectiva
		\item \textbf{Isomorfismo:} si $\varphi$ es monomorfismo y epimorfismo al mismo timpo. 
    	\end{itemize}
\end{enumerate}				
% Fin de definición de homomorfismo de grupos y biyecciones
\end{df}
Un isomorfismo tiene como función una inversa, esta función inversa  cumpe ser isomorfismo tambén. De esta manera podemos pensar a los grupos salvo isomorfismo cuando sea posible, en otras palabras, en analogía a espacios topológicos queremos estudiar a un grupo por medio de otro conocido mediante algún isomorfismo. 

\begin{ej}
Sea $G$ un grupo generado por un solo elemento $g$ de orden finito $n$ entonces $G$ es isomorfo al grupo de los enteros modulo $n$, denotado usualmente por $\Ent_n = \Ent/n \Ent$. Un isomorfismo esta dado por

\begin{align*}
\varphi:G \to \Ent_n \\
g^k \mapsto [k],
\end{align*}
esto es a cada elemento de $G$ que se escribe como pontencia de  $g$ se le asocia la clase $[k]$ en $\Ent/\Ent_n$. Más aún, si $G$ es inifitamene generado entonces 
\begin{align*}
\varphi:G \to \Ent \\
g^k \mapsto k,
\end{align*}
es un isomorfismo.
\end{ej}

Una manera de construir nuevos grupos por grupos establecidos es por medio de los cocientes, el siguiente resultado es muy importante no solo por que nos ayuda a entender la teoría de Galois, nos da una manera de construir nuevos grupos y puede consultarse en \cite{alg_zald}
	
\begin{te}
% Teorema del grupo cociente.
Sea $H$ subrupo normal de  $G$ entones el conjunto $G/H$ tiene una estructura un grupo y hace de la proyección $\pi:G \to G/H$ un homomorfismo de grupos.
\end{te}
	
La normalidad es muy importante en el teorema anterior. Una parte de la teoría de cubrientes nos permite definir el siguiente homomorfismo de grupos.


\begin{ej}
Del ejemplo \ref{ej:Cir_un_coc} consideremos a $S^1=\mathbb{R} / \Ent$ y al subgrupo de homeomorfismos que preservan orientación que vamos a de notarlo por $\Hom^+(S^1)$. De la teoría de cubrientes para $f: S^1 \to S^1$ un homeomorfismo que preserva la orientación se tiene que existe una función $\tilde{f}:\mathbb{R} \to \mathbb{R}$ tal que $\pi(\tilde{f})=f$, esto es $f([x])=[\tilde{f}(x)]$ y que satisface la relación,

$$\tilde{f}(x+1)=\tilde{f}(x)+1.$$
Tenemos un morfismo de grupos $\varphi: Hom_+(S^1) \to Hom(S^1)$. 
Recíprocamente toda función  $\tilde{f}:\mathbb{R} \to \mathbb{R}$ estrictamente  creciente que satisface la relación, $\tilde{f}(x+1)=\tilde{f}(x)+1,$ induce un homeomorfismo en $S^1.$ Finalemente concluimos con la observación, $\varphi$ es un epimorfismo de grupos. 
\end{ej}
	
	Los siguientes términos son importantes, tanto desde la historia del estudio de los grupos, como para definir a las acciones sobre conjuntos.	
	

	\begin{df}
	Sea $X$ un conjunto no vacío y $G$ grupo, una \textbf{acción} es una función $\alpha:G \times X \to X$ que satisface,
	\begin{enumerate}
	\item $\alpha(e,x)=x$
	\item $\alpha(g,\alpha(h,x))=\alpha(gh,x)$
	\end{enumerate}
Si en un conjunto $X$ se tiene una acción de un grupo $G$ decimos que $X$ es un $G$-\textbf{conjunto}.
	\end{df}

\begin{ej}
Sean $G$ un grupo y $H$ un subgrupo. $H$ actua sobre $G$ por la acción $*:H \times G \to G$ dada por $*(h,g)=hg$, una traslación a la izquierda.  Más aún si $H$ es normal en $G$ acuta sobre $G/H$ mediante $*:G \times G/H \to G/H$ dada por $*(g,tH)=(gt)H.$ 
 \end{ej}
%%%%%%%%%%%% conmutadores, grupos derivados %%%%%%
\addcontentsline{toc}{section}{Grupos conmutadores, grupos derivados}

\section*{Grupos conmutadores, grupos derivados}
Para este trabajo usaremos a los conjugados de un elemento y los conmutadores. Para ello vamos a estudiar un poco de notación y de teoría para acortar nuestras demostraciones. 

\begin{df}
Sea $G$ un grupo y $g$, $h \in G$. Definimos al \textbf{conmutador} de $g$ y $h$ como el elemento 

\begin{align*}
[g,h]=ghg^{-1}h^{-1}.
\end{align*}

Dados dos subconjutos $H$ y $K$ de $G$, definimos al grupo $[H,K]$ como el grupo generado por los elementos $\{[h,k]:h \in H \text{ y }k \in K \in G \}$. En particular al grupo $[G,G]$ le llamaremos el \textbf{primer grupo derivado} de $G$ y vamos a denotarlo por $G'$ . Además, cuando $ G' = G $ diremos que el grupo $G$ es \textbf{perfecto}, en este caso, todo elemento de $G$ es producto de un numero finito de conmutadores. 
\end{df}
	
\begin{ob}\label{ob:pr_de_los_conmutadores}
Haremos unas observaciones de la definición anterior. Sean $G$ un grupo y $g$, $x$, $y \in G$ con $[x,y]$ su conmutador.

\begin{enumerate}
	\item El inverso de un conmutador,
	\begin{align*}
	[x,y]^{-1} & =(xyx^{-1}y^{-1})^{-1} \\
	& = (yx^{-1}y^{-1})^{-1}x^{-1} \\
	& = (x^{-1}y^{-1})^{-1}y^{-1}x^{-1} \\
	& = (y^{-1})^{-1}xy^{-1}x^{-1} \\
	& = yxy^{-1}x^{-1} = [y,x] .
	\end{align*}
	
	\item El conjugado de un conmutador, 
	\begin{align*}
	[x,y]^{g} & = g(xyx^{-1}y^{-1})g^{-1} \\
	& = (gxg^{-1})(gyg^{-1})(gx^{-1}g^{-1})(gy^{-1}g^{-1}) \\
	& =  [y^{g},x^{g}].
	\end{align*}
	
	\item El conmutador bajo homeomorfismos. Sea $\phi:G \to G$ un morfismo de grupos, notemos que,
	\begin{align*}
	\phi([x,y]) & = \phi(xyx^{-1}y^{-1})\\
	& = \phi(x)\phi(y)\phi(x^{-1})\phi(y^{-1}) \\
	& = \phi(x)\phi(y)\phi(x)^{-1}\phi(y)^{-1} \\
	& =  [\phi(x),\phi(y)],     
	\end{align*}
	más aún sea $x \in [G,G]$ es decir existen $a_i$, $b_i \in G$, $i=1, \cdots n$ tales que
	\begin{align*}
	x=[a_1,b_1]^{m_1} \cdots[a_n,b_n]^{m_n},
	\end{align*}
	notemos que para todo morfismo de grupos, $\phi$ se cumple que
	\begin{align*}
	\phi(x)=[\phi(a_1),\phi(b_1)]^{m_1} \cdots[\phi(a_n),\phi(b_n)]^{m_n} \in [G,G].
	\end{align*}
	 Por tanto $\phi([G,G]) \subset [G,G]$.
	\item $[G,G]$ es normal en $G$. Sea $g \in G$ y consideremos el automorfismo interno 
	
	\begin{align*}
	\gamma_g:G \to G \\
	x \mapsto x^{g}
	\end{align*}	 
	
	claramente se tiene que $\gamma_g([G,G])=g[G,G]g^{-1}$ y por el inciso anterior $g[G,G]g^{-1}=\gamma_g([G,G]) \subset [G,G]$, concluimos así la normalidad.
\end{enumerate}
\end{ob}

Una relación interesante entre conjuntos generador y el derivado del grupo generado. 
\begin{pr} \label{pr:Derivados_y_generadores}

Sea $G$ un grupo  generado por un conjunto balanceado $X$ entonces para $G'$ el primer subgrupo conmutador es generado por conjugados de conmutadores de elementos de $X$. Esto es
$$G'=\langle\{ [a,b]^{g}: g \in G \text{ y } a, b \in X \} \rangle$$
\end{pr}

\begin{proof}

Sea $W=\langle\{ [a,b]^{g} : g \in G \text{ y } x,y \in X \}$. Notemos que para $a$, $b \in X$ y $g \in G$ de la observación anterior tenemos que,

$$[x,y]^{g}=[x^{g},y^{g}]$$

$G'$ es un grupo que contiene al conjunto generador de $W$ por tanto $W \subset  G'.$ Para la otra contención lo haremos por casos esto para facilitar la complejidad del argumento final. 

Sean $a, b \in G$ y supongamos el caso en que $b \in X$. Estudiaremos el resultado por la cantidad de factores de  $a$. Como $X$ genera a $G$, existen $s_1 \cdots s_n \in X$ tales que $a= s_1^{\alpha_1} \cdots s_n^{\alpha_n}$ con $\alpha_i \in \{1, -1\}$, notemos que $(s^{-1})^1=s^{-1}$ entonces para hacer sencilla la escritura sin pérdida de generalidad vamos escribir  $\alpha_i=1$ para $i \in \{1, \cdots, n\}$.  Por inducción fuerte sobre $n$ veremos el resultado, notemos que en el caso $n=1$, el conmutador de $a$ y $b$ es tal que,

\begin{align*}
[a,b]=[s_1,b] \in W
\end{align*}

y para el caso $n=2$,

\begin{align*}
[s_1s_2,b] & =(s_1s_2)b(s_2^{-1}s_1^{-1})b^{-1} \\
& = s_1(s_2bs_2b^{-1})(bs_1^{-1}b^{-1}s_1)s_1^{-1}\\
& = s_1([s_2,b])s_1^{-1}s_1([b,s_1^{-1}])s_1^{-1}\\
& = ([s_2,b]^{s_1})([b,s_1^{-1}]^{s_1})
\end{align*}
notemos que $s_1$ como exponente es un conjugado y no como potencia del elemento. De lo anterior $[a,b]$ es un producto de conjugados de elementos de $X$ por 
tanto $[a,b] \in W.$ Para el caso $n=3$ tenemos que 

\begin{align*}
[s_1s_2s_3,b] & =(s_1s_2s_3)b(s_3s_2s_1)b^{-1} \\
& = s_1(s_2s_3 b s_3^{-1} s_2^{-1} b^{-1})(bs_1b^{-1}s_1^{-1})s_1 \\ 
& = s_1 [s_2 s_3 ,b][b,s_1]s_1\\
& =([s_2 s_3,b]^{s_1})([b,s_1]^{s_1})
\end{align*}

De la primera parte para $s_2 s_3$ se tiene que,

\begin{align*}
[s_1s_2s_3,b] & = ([s_3,b]^{s_2})([b,s_2^{-1}]^{s_2})([b,s_1^{-1}]^{s_1})
\end{align*}

nuevamente tenemos que $[a,b] \in W.$ Por inducción fuerte, supongamos que se cumple para $1, \cdots, n-1$ veremos el caso $n$,

\begin{align*}
[s_1 \cdots s_n,b] & =(s_1 \cdots s_n)b(s_n \cdots s_1)^{-1} b^{-1} \\
& = (s_1 \cdots s_n)b(s_n^{-1} \cdots s_2^{-1 }b^{-1}b s_1^{-1})b^{-1}(s_1 s_1^{-1}) \\
& = s_1( s_2\cdots s_n )b(s_n^{-1}  \cdots s_2^{-1} )b^{-1}(b s_1^{-1}  b^{-1}s_1) s_1^{-1}  \\
& = ([ s_2 \cdots s_n,b]^{s_1})([b,s_1^{-1} ]^{s_1})
\end{align*}



por hipótesis de inducción tenemos que $[ s_2 \cdots s_n,b]$ es producto de conjugados de conmutadores de $X$, a saber

\begin{align*}\label{eq:induccion}
[a,b]=[s_1 \cdots s_n , b]  = & [ s_n , b] ^{(s_1 \cdots s_{n-1})}[b,s_{n-1}^{-1}]^{(s_1  \cdots s_{n-1})} * \\ 
& [b,s_{n-2}^{-1}]^{(s_1 \cdots s_{n-2})} [b,s_{n-3}^{-1}]^{(s_1  \cdots s_{n-3} )} \cdots [b,s_1^{-1}]^{s_1}
\end{align*}

Por tanto tenemos que cada factor es un conjugado  de un conmutador de elementos de $X$ tenemos que $[a,b] \in W$ y terminamos la hipótesis de inducción. En general si $b\not\in X$, se tiene que $b=t_1 \cdots t_m$ donde cada $t_i \in X$. Notemos que de la inducción obtuvimos la expresión 

\begin{align*}
[a, b]  = & [ s_n , b] ^{(s_1 \cdots s_{n-1})}[b,s_{n-1}^{-1}]^{(s_1  \cdots s_{n-1})} * \\ 
& [b,s_{n-2}^{-1}]^{(s_1 \cdots s_{n-2})} [b,s_{n-3}^{-1}]^{(s_1  \cdots s_{n-3} )} \cdots [b,s_1^{-1}]^{s_1}
\end{align*}

 donde los términos $[s_n,b]$ y $[b,s_j]$ cumplen que $s_1, \dots , s_n \in X$. Notemos que $[b,s_j^{-1}]=[t_1 \cdots t_m,s_j^{-1}]$ se reducen al caso que demostramos anteriormente y notemos que $[s_n,b]^{-1}=[t_1 \cdots t_n,s_n]$ aplica el mismo argumento, como $W$ es grupo $[s_n,b]  \in W$  entonces $[a,b]$ se factoriza como producto de elementos de $W$. Por tanto $W$ contiene a todos los conmutadores de $G$, como $G'$ es el minimo grupo que contiene a ese conjunto se tiene que $G' \subset W.$ Dando así la igualdad.

\end{proof}

%%%%%%%%%%%%%%%%%%%%%%%%%%%%%%% Grupos topológicos

\addcontentsline{toc}{section}{Grupos topológicos}
\section*{Grupos topológicos}
La estructura matemática que resulta  de considerar una topología en un grupo es muy interesante, para nuestros fines no podremos profundizar en este tema, simplemente hacemos mención acerca de ciertas propiedades de grupos topológicos que usaremos.  

\begin{df}
	Sea $X$ un grupo decimos que $X$ es un \textbf{grupo topológico} si existe una topología en $X$ de manera que las funciones 
	\begin{enumerate}
	\item  Multiplicación,
	\begin{align*}
	\circ : X \times X & \to X \\
	(x,y) & \mapsto xy
	\end{align*}
	
	\item Inversión
	\begin{align*}
	^{-1} : X & \to X \\
	x & \mapsto x^{-1}
	\end{align*}		
	 
	\end{enumerate}
son continuas.	
	\end{df}

 \subsection*{Propiedades topológicas de un grupo topológico}
En esta sección vamos a demostrar unos resultados que vamos  a usar y que son omitidos (quizás a propósito) por los autores de \cite{ander} y \cite{epst}. Consideramos a estos resultados un paso de formalidad completo a los artículos citados. 

\begin{ob}
	Sea $G$ un grupo topológico.
	
	\begin{enumerate}
	\item Sea $Q(e)$ la componente conexa de la identidad. Como la función multiplicación es continua se sigue que, para toda $h \in G$ el conjunto $hQ(e)$ es conexo y contiene al elemento $h$.

 \item $g \in Q(e)$ si y solo si $e \in Q(g).$ En particular $Q(g)=Q(e)$.
 
 \item  Si $h \in Q(e)$ entonces $e \in Q(h^{-1})$. En particular $h \in Q(e)$ si y solo si $h^{-1}\in Q(e)$.
	\end{enumerate}
 \end{ob}	
 
 \begin{proof}
 Sea $g \in Q(e)$, notemos que $Q(e)$ es un conjunto conexo que tiene a $g$ por tanto $Q(e) \subset Q(g)$, en particular $e \in Q(g).$ El recíproco es idéntico y se omite. Para la otra parte, notemos que $h^{-1}Q(e)$ es un conjunto conexo que tiene a $h^{-1}$ por tanto $h^{-1}Q(e) \subset Q(h^{-1})$, como $h \in Q(e)$ se sigue que el elemento,

$$h^{-1}h \in h^{-1}Q(e)$$

y así $e \in Q(h^{-1})$.  
\end{proof}

\begin{lm}\label{lm:Q(e) es grupo}
La componente conexa de la identidad es un grupo.
\end{lm}


\begin{proof}
 Veremos que el conjunto $Q(e)$ es cerrado bajo la operación de $G$. Sean $g$, $h \in Q(e)$, por la observación previa resta ver que $e \in Q(gh)$. Tenemos que $ ghQ(e) \subset gQ(h) \subset Q(gh)$ por otro lado notemos que, 
 \begin{align*}
 gh(h^{-1}) \in  ghQ(e)
 \end{align*}
 y así $g \in ghQ(e)$, por tanto tenemos la contención $ghQ(e) \subset Q(g)=Q(e)$, concluimos que $gh \in Q(e).$
 \end{proof}	
	
	
	
\begin{lm}\label{lm:gU_es_abierto}
Sea $(X,\tau, \circ)$ grupo topológico y $U$ abierto en $X$, para todo $h \in X$ el conjunto $hU$ es abierto.
\end{lm}

\begin{proof}
Sea $h$ en $X$, las siguientes funciones

\begin{align*}
\varphi_h:X \to X \\
g \mapsto hg
\end{align*}
y 
\begin{align*}
\phi_h:X \to X \\
g \mapsto h^{-1}g
\end{align*}
 son continuas e inversas una de otra. Resta notar que $\varphi_h(U)=hU$ y  la imagen directa de $\varphi$ es la imagen inversa de su función inversa  $\varphi_h(U)=\phi_h^{-1}(U)$ que por continuidad es un conjunto abierto.  Tenemos así que $hU$ es un conjunto abierto.

\end{proof}

% Una vecindad de la identidad genera un grupo conexo. 
\begin{pr} \label{pr:vec_de_la_id_gen}
Sea $(X, \tau, \circ)$ un grupo topológico conexo. Para toda vecindad $U$ de $e$ se cumple que $\langle U\rangle=X.$
\end{pr}
	
\begin{proof}
Sea $U \in \mathcal{N}_e$, resta ver que $G \subset \langle U \rangle$. Para ello veremos que $\langle U \rangle$ es un conjunto abierto y cerrado en $X$ y usaremos la conexidad de $X$ para concluir la igualdad.

 Sea $g \in \langle U \rangle$, por definición de subgrupo generado, para todo subgrupo $H$ de $X$ que contiene a $U$ se cumple  que 
 
\begin{enumerate}
	\item $g \in H$,
	\item al ser cerrado como subgrupo tenemos que, para toda $u \in U$ el elemento $gu$ está en $H$, por tanto $gU \subset H$.
 
 \item Por el lema \ref{lm:gU_es_abierto} el conjunto $gU$ es abierto.
\end{enumerate} 
 
 
Más aún, $g=ge \in gU$, de esta manera se que tiene que  $gU$ es vecindad de $g$ y junto con la contención $gU \subset H$ de la definición de grupo generado tenemos que  $gU \subset \langle U \rangle$ y por tanto $\langle U \rangle$ es un conjunto abierto.

Para ver que $\langle U \rangle$ es cerrado, sea $h \in \overline{\langle U \rangle}$ y consideremos al conjunto $hU$ que, por el lema \ref{lm:gU_es_abierto} es abierto y en particular es vecindad de $h$ y por definición del conjunto cerradura tenemos que, $hU \cap \langle U \rangle \neq \emptyset.$ De esta manera, sea $g \in hU\cap \langle U \rangle$ en particular, como $g \in hU$ existe $u \in U$ tal que $g=hu$, consideremos lo siguiente, $h=gu^{-1} \in  \langle U \rangle U =\langle U \rangle$ por tanto $\overline{\langle U \rangle} \subset \langle U \rangle.$ Tenemos que $\langle U \rangle$ es un conjunto cerrado y abierto en un espacio conexo, entonces o $\langle U \rangle=\emptyset$ o $\langle U \rangle=X$ como $e \in \langle U \rangle$ concluimos que $\langle U \rangle = X.$
\end{proof}

%%%%%%%%%%%%%%%%%%%%%%%%%%%%%%%%%%%%%%%%%%%%%%%%%%%%%%%%%
\addcontentsline{toc}{section}{Topología compacto-abierta}
\section*{Topología compacto-abierta}	
Tomamos el resultado siguiente de \cite{top_juan} páginas 105-110. Sean $X,$ $Y$ espacios topológicos, $Y^X$ la familia de funciones $f:X \to Y$, para cada $A \subset X$ y $B \subset Y$ denotamos a $$\mathcal{U}(A,B)=\{f \in Y^X|f(A)\subset B \}$$ en caso de que $A=\{a\}$ denotamos  simplemente por $\mathcal{U}(a,B)=\{f \in Y^X|f(a) \in B\}$.

%\cite{} pág 107 topo T18



\begin{pr}
	La colección de las familias de la forma $\mathcal{U}(K,B)$ donde $K$ es un compacto de $X$ y $U$un abierto de $Y$ es una subbase para una topología en $Y^X$. Dicha topología será llamada \textbf{topología compacto-abierta}.
\end{pr}

\begin{ob}
Un hecho interesante es que la topología compacto abierta es mas fina que la topología producto y es mas gruesa que la topología caja, pero en el caso de un producto finito la tres topologías coinciden.
\end{ob}

\begin{df}
Sea $X_d$ un espacio métrico compacto. Definimos a la métrica de la convergencia uniforme como;

$$d(f,g)= \max_{x \in X}\{d(f(x),g(x)) \}$$
donde $f$ y $g$ son funciones continuas. 
\end{df}


Ahora vamos a evitar desarrollar una parte de la teoría. Queremos solo usar una equivalencia entre la topología compacto abierta y la generada por la métrica uniforme. Para ello sugerimos revisar \cite{top_munk} página 321 hasta 325, también \cite{top_willd} desde las páginas 278 hasta 284. Para revisar la referencia de Willard, de tener en cuenta el capítulo 10, el libro de Munkres es adecuado para una introducción al tema y el libro de Willard es un libro detallado.

\begin{df}
Sean $(Y,d)$ un espacio métrico y $X$ un espacio topológico. Sea $f \in Y^X$, $C$ subespacio compacto de $X$ y $\varepsilon > 0$. Definimos a $B_C(f, \varepsilon)$ como la familia de funciones $g \in Y ^X$ para las cuales,

$$\sup \{d(f(x),g(x))|x \in C \} < \varepsilon.$$

Las familias  $B_C(f,\varepsilon)$ forman una base para una topología sobre $Y^X$ la cual será llamada la \textbf{topología de la convergencia compacta}.  
\end{df}
El siguiente resultado puede consultarse en \cite{top_willd} página 284 o \cite{top_munk} página 325.

\begin{te}
Para un espacio de funciones, $X^Y$ donde $X$ es compacto, la topología de la convergencia compacta es la topología compacto abierta. 
\end{te}


El siguiente teorema puede encontrarse en \cite{top_munk} como teorema 46.7

\begin{te}
Sean $X$ un espacio topológico e $Y_d $ métrico. Para el espacio de funciones $Y^X$ se tiene la siguiente inclusión de topologías:

$$\text{(uniforme)} \subset \text{(convergencia compacta)}$$ 

Si $X$ es compacto entonces las topologías coinciden. 
\end{te}

Este teorema puede estudiarse en \cite{top_munk} como teorema 46.8 o en \cite{top_willd} como theorem 43.7

\begin{te}
Sean $X$ un espacio topológico e $Y_d $ métrico. Sobre el subespacio de funciones continuas de $X$ a $Y$ se tiene que la topología de la convergencia compacta y de la convergencia uniforme coinciden. 
\end{te}

Continuamos con la homogeneidad de un espacio topológico. 

%%%%%%%%%%%%%%%%%%%%%%%%%%%%%%%%%%%%%%%%%%%%%%%%%%
\addcontentsline{toc}{section}{Homogeneidad de espacios topológicos}
\section*{Homogeneidad de espacios topológicos}
Sea $X$ un espacio topológico, decimos que $X$ es homogéneo entre dos puntos distintos $x$ e $y$ si existe un homeomorfismo,  $h:X \to X$ tal que $h(x)=y$.


\begin{nt} $\Hom(X)$ denotará el grupo de homeomorfismos $h:X \to X$ y la función identidad $\Ide_X:X \to X$ denotará el neutro de $\Hom(X)$. Notemos que $\Hom(X)$ es un grupo con la composición de funciones.
	\end{nt}
	
		\begin{proof}
	El resultado es directo que la composición de funciones biyectivas es biyectiva y que composición de funciones continuas es continua. 
\end{proof}
	
\begin{df}% Conjunto de soporte
Sea $K \subset X$ y $h:X \to X$ un homeomorfismo. Decimos que $K$ es un  \textbf{ conjunto de soporte} para $h$ (o que $h$ está soportado en $K$) si,

\begin{enumerate}
	\item $X \setminus K$ es no vacío y
	\item $h|_{X \setminus K}=\Ide|_{X \setminus K}$.
\end{enumerate}	
 Al conjunto $ \Sop(h)= \Cla\left( \{x \in X : h(x)\neq x \} \right),$  le llamaremos el \textbf{soporte} de $h$. Si $h$ no es la función identidad entonces $\Sop(h)$ es no vacío y para $K$ un conjunto de soporte para $h$ se tiene que
 
\begin{align*}
     \Sop(h) \cap (X \setminus K) = \emptyset
\end{align*}
 por tanto $\Sop(h) \subset K $ el conjunto $\Sop(h)$ es un conjunto mínimo de los conjuntos de soporte para una homeomorfismo.  A la familia de homeomorfismos, $g:X \to X$, para los cuales existe $U \in \tau$ tal que $g|_U=\Ide|_U$ le denotaremos por $\Hom^0(X)$.
\end{df}
 

\begin{ob} \label{ob:sop_fun_inversa}
\begin{enumerate}
Sea $g:X \to X$ un homeomorfismo que está soportado en $K$.
    \item Como la función $g^{-1}$ es una función biyectiva tenemos que, 
    \begin{align*}
        \{x \in X : g(x) \neq x \}  = \{x \in X : x \neq g^{-1}(x) \}         
    \end{align*}
    
    Al tomar cerradura, se tiene que $\Sop(g) = \Sop(g^{-1})$,  $g$ y $g^{-1}$ tienen el mismo soporte. 
     
    \item Sea $K$ un conjunto de soporte para $g$. Para la función inversa $g^{-1}:g(K) \to K$ tenemos que,

\begin{align*}
\Ide|_{X \setminus K}=g|_{X \setminus K},
\end{align*}

como la función identidad es su propia inversa tenemos que 

$$g^{-1}|_{X \setminus K}=g|_{X \setminus K},$$
 del inciso anterior tenemos que $\Sop(g^{-1}) \subset K$ tenemos que $g^{-1}$ está soportado en $K$. Además notemos que
 
\begin{align*}
K \cap (X \setminus g(K))=K \cap g(X \setminus K)=K \cap X \setminus K = \emptyset
\end{align*}
por tanto $K \subset g(K)$. Por otro lado

\begin{align*}
g(K) \cap (X \setminus K)=g(K) \cap g(X \setminus K)=g(K \cap X \setminus K) = \emptyset
\end{align*}
tenemos que $K=g(K)$.
\item Si $f$ y $g$ son dos funciones con soportadas en $K$ entonces $fg(K)=K=gf(K).$ Por inducción esto se tiene para una cantidad finita es decir, $f_1 \cdots f_n$ entonces $f_{\sigma(1)} \cdots f_{\sigma(n)}(K)=K$ para $\sigma$ una permutación de índices.   
\end{enumerate}

\end{ob}

\begin{ob}\label{ob:efecto_soporte}
La composición de funciones con soportes ajenos. Sean $f$, $g$ dos homemorfismos tales que $K_f$ y $K_g$ son conjuntos de soporte respectivamente y tales que  $K_f \cap K_g = \emptyset$ y $K_f \cup K_g \neq X$ entonces para $h=fg$ se cumple que, $h=g$ en  $ K_g$, $h=K_f$ en $K_f$ y $h$ tiene soporte en $K_f \cup K_g$. Esto se puede hacer para una cantidad finita de pares de funciones $f_ng_n.$  Para la primera parte, notemos que  $h(K_f)=fg(K_f)$, como $g(K_f) \subset g(X \setminus K_g)$ tenemos que $g|_{K_f}=Id$, por tanto $h|_{K_f}=f|_{K_f}$. El resultado es análogo a para $h|_{K_g}=g|_{K_g}$. Para la última parte, notemos que 
 
\begin{align*}
h( X \setminus (K_f \cup K_g)= h(X \setminus K_f) \cap h(X \setminus K_g))
\end{align*} 

de la primera parte tenemos que, dado $x \in (X \setminus K_f) \cap (X \setminus K_g)$

\begin{align*}
fg(x)= f(x)=x.
\end{align*} 
Por tanto, se tiene que $h$ está soportado en $K_f \cup K_g.$ De manera inductiva supongamos que $(f_i)_{i = 1}^n$ y $(g_i)_{i = 1}^n$ son familias de funciones tales que están soportadas en $K$ y $W$ respectivamente. Para $h = g_1f_1 \cdots g_n f_n$ se cumple que $h= g_1 \cdots g_n$ en $W$ y $h=f_1 \cdots f_n$ en $K.$ Notemos que, 
 
\begin{align*}
h(K)= g_1f_1 \cdots g_n f_n(K)
\end{align*}
 se tiene que $f_i(K)= K \subset X \setminus W$ para $i = 1 , \cdots ,n$ por tanto para $$g_n(f_n(K))=f_n(K),$$
 
 al componer por $g_{n-1}f_{n-1}$ y  aplicando la observación \ref{ob:sop_fun_inversa}
 $$g_{n-1}f_{n-1}g_n(f_n(K))=f_{n-1}f_n(K),$$

repitiendo este proceso una cantidad finita de veces, se tiene que

 $$h(K)=f_1 \cdots f_{n-1}f_n(K).$$
Para el caso en que $h(W)= g_1 \cdots g_{n-1}g_n(W)$ es similar.  Además si $x \in X (\setminus K) \cap (X \setminus W)$ entonces de la primera parte de esta observación,

\begin{align*}
h(x) & = g_1f_1 \cdots g_n f_n(x) = g_1f_1 \cdots g_{n-1 }f_{n-1}(x) \cdots = x
\end{align*}

y $h$ está soportado en $K \cup W.$
\end{ob}
 
 
 La observación anterior es una manera de generalizar a la propiedad $(E)$ de \cite{ander}. El siguiente lema será usado en las secciones posteriores, es importante, no solo nos deja manipular el soporte de un homeomorfismo bajo una conjugación sino que nos da la herramienta importante sobre ciertos conmutadores de homeomorfismos para el teorema 1 del trabajo de Anderson en \cite{ander}, también es importante mencionar que Epstein en \cite{epst} tambien lo usa. Este lema se encuentra en \cite{ander} como property (A).

\begin{lm}\label{lm:obs_A}
Sean $K \subset X$ y $g \in \Hom^0(X)$ soportado en $K$. Para cualquier $h \in \Hom(X)$ se tiene que,
 
	\begin{enumerate}
	\item  $g^{h^{-1}}$ está soportado en $h^{-1}(K)$.
	\item Si $h^{-1}(K) \cap K = \emptyset$ y $h^{-1}(K) \cup K \neq X$  entonces
		\begin{enumerate}
		\item $[h^{-1},g^{-1}]$ está soportado en $h^{-1}(K) \cup K$,
		\item $[h^{-1},g^{-1}]|_K=g|_K$ y 
		\item $[h^{-1},g^{-1}]|_{h^{-1}(K)}=h^{-1}g^{-1}h|_{h^{-1}(K)}$
		\end{enumerate}	
	\end{enumerate}
\end{lm}
	
\begin{proof}
Para el primer inciso. Sea $x \in X \setminus h^{-1}(K)= h^{-1}( X \setminus K)$ de donde $h(x) \in X \setminus K$, como $K$  es un conjunto de soporte para $g$ tenemos que, $g(h(x))=h(x)$ de esta manera al componer con la función $h^{-1}$ por la izquierda obtenemos lo siguiente, $h^{-1}gh(x)=x$ y tenemos que, 
 
 $$g^{h}(x)|_{X \setminus h^{-1}(K)}=\Ide_{X \setminus h^{-1}(K)},$$ 
 
por definición concluimos que $h^{-1}gh(x)$ está soportado en $h^{-1}(K).$ Ahora veremos el segundo inciso. De la primera parte tenemos que $h^{-1}g^{-1}h$ tiene soporte en $h^{-1}(K)$ y $g$ en $K$ los cuales son conjuntos ajenos por tanto de la observación observación \ref{ob:sop_fun_inversa}, la observación anterior, tenemos que $[h^{-1},g^{-1}]$ tiene soporte en $K \cup h^{-1}(K).$
   
\end{proof}


Finalizamos este capitulo recordando que nuestra introducción pudiera no abarcar todos los resultados que vamos a mencionar, sin embargo mencionamos los libros que hemos consultado donde pudiera  estar la demostración detallada  o bien un estudio profundo de dicho tema.


\addcontentsline{toc}{chapter}{Simplicidad del grupo $\Hom(S^1)$}
\chapter*{Simplicidad del grupo $\Hom(S^1)$}

 En 1999 E. Ghys en \cite{ghys} dio una adaptación del trabajo de Epstein \cite{epst} para el grupo de homeomorfismos que preservan orientación para demostrar que el grupo de homeomorfismos del círculo es simple. Epstein en \cite{epst} trabajó en variedades, espacios topológicos muy parecidos a un espacio vectorial $\mathbb{R}^m$ sugerimos para ese tema la bibliografía dada en \cite{palmas}.  Recordemos al circulo unitario como al espacio $\{x \in\mathbb{R}^2:|x|=1 \}$, denotado por $S^1$, el círculo unitario como un subespacio del plano $\mathbb{R}^2$ sin embargo como vimos anteriormente, este espacio es homeomorfo a $\Rea / \Ent$ como en el ejemplo \ref{ej:Cir_un_coc} más aún, a cualquier espacio homeomorfo a $S^1$ le llamaremos curva cerrada simple.
 
\begin{cn}
En esta sección vamos hacer el abuso de notación.
	\begin{itemize}
	\item A partir de ahora y lo que resta de la sección, $\Hom(S^1)$ denotará al grupo $\{h: S^1 \to S^1 | h \text{ es homeomorfismo y preserva orientación} \}$. Por preservar la orientación nos referimos como en el ejemplo \ref{ej:Cir_un_coc}.
	\item Un conjunto es no degerado si tiene mas de un punto. A los subconjuntos cerrados conexos y no degenerados de  $S^1$ serán llamados \textbf{sub-intervalos} de $S^1.$ 
	\end{itemize}
\end{cn}

Veremos ahora que el grupo de homeomorfismos del círculo es un grupo topológico con la topología compacto abierta (o métricas como según nos convenga) y la composición de funciones como operación de grupo.

\addcontentsline{toc}{section}{El grupo topológico $\Hom(S^1)$}
\section*{El grupo topológico $\Hom(S^1)$}

 Estudiaremos la topología compacto abierta en el espacio $S^1$ con el fin de ver que la terna $(\Hom(S^1),\circ, \tau_{CA})$ es un grupo topológico. Para demostrar que operación inversión es continua usaremos la equivalencia métrica. Sea 
 
\begin{align*}
^{-1}:S^1 \to S^1 \\
g \mapsto g^{-1},
\end{align*} 

la funcón inversión de $S^1$. Sean $\varepsilon >0$ y  $h,$ $g \in \Hom(S^1)$. Veremos que existe $\delta >0$ tal que si $d(g,h) < \delta$  se implica que $d(g^{-1},h^{-1}) < \varepsilon,$ en la métrica de la convergencia uniforme. Observemos que; si $g$ es una función continua en un espacio compacto entonces $g$ es uniformemente continua, además si $g$ es un homeomorfismo entonces la función $g^{-1}$ es uniformemente continua por tanto, de la continuidad uniforme de $g^{-1}$ se tiene que, para todo $x \in S^1$ existe $\delta >0$ tal que si $|x-y|< \delta$ entonces,

$$\|g^{-1}(x)-g^{-1}(y)\| < \varepsilon.$$

Consideremos la bola $B_d(h,\delta)$ y $x_0$ fijo. Como $h$ es biyectiva existe $z_0 \in S^1$ tal que $z= h^{-1}(x)$, por la definición de la métrica uniforme se cumple que,

$$\|h(z_0)-g(z_0)\| \leq d(h,g) < \delta $$

por tanto, de la continuidad de $g^{-1}$ se sigue que,

$$|g^{-1}(h(z_0))-g^{-1}(g(z_0))| < \varepsilon,$$

y notemos que 

$$ |g^{-1}(x_0)-h^{-1}(x_0)|=|g^{-1}(h(z_0))-z_0| < \varepsilon. $$

Luego, tomando el máximo tenemos que $d(g,h) < \varepsilon,$ de esta manera la función inversión es un  función continua. Veremos ahora la continuidad de la multiplicación mediante la topología compacto-abierta. Antes, hacemos mención de un resultado que no demostraremos pero que nos será de utilidad y puede encontrarse en \cite{top_willd} como Lemma 43.3.

\begin{lm}
En un espacio regular, si un conjunto $F$ es compacto, $U$ es un abierto y $F \subset U$ entonces existe un conjunto abierto $V$ tal que $F \subset V \subset \overline{V} \subset U.$
\end{lm}

Afirmamos que la operación composición

\begin{align*}
\circ:\Hom(S^1) \times \Hom(S^1) & \to \Hom(S^1) \\
(g,h) & \mapsto gh,
\end{align*} 

es continua en la topología compacto-abierta. Sean $g,$ $h \in \Hom(S^1)$ y consideremos un abierto básico en la topología compacto abierta que sea vecindad de $gh$, explícitamente sean $K$ compacto de $S^1$ y $U$ abierto en $S^1$ tal que

$$gh \in\mathcal{U}(K,U)=\{f \in \Hom(S^1): f(K) \subset U\}$$

 es claro que $gh(K) \subset U$ y como $g$ es biyectiva,  componiendo con la función $g^{-1}$ se tiene que; $h(K) \subset g^{-1}(U),$ notemos que $g$ y $h$ son homeomorfismos por tanto $h(K)$ es compacto y $g^{-1}(U)$ es abierto, como $S^1$ es un espacio normal, existe $V$ abierto en $S^1$ tal que 

$$h(K) \subset V \subset \overline{V} \subset g^{-1}(U),$$
vamos a considerar al básico de   $\Hom(S^1) \times Hom(S^1)$ dado por  
$$\mathcal{W}:=\mathcal{U}(\overline{V},U) \times \mathcal{U}(K,V).$$

Como $\overline{V} \subset g^{-1}(U)$ componiendo con la función $g$ se sigue que $g(\overline{V}) \subset U$ y como $h(K) \subset V$ tenemos que $(g,h) \in \mathcal{W}$. Veremos que la imagen de $\mathcal{W}$ bajo la operación $\circ$ está contenida en $\mathcal{U}(K,U)$. Sea $(f_1,f_2) \in \mathcal{W}$, tenemos las siguientes contenciones,  

\begin{align*}
f_1(\overline{V})\subset U \\
f_2(K)\subset V 
\end{align*}

componiendo la primera contención por $f_1^{-1}$ se sigue que

$$f_2(K) \subset V \subset \overline{V} \subset f_1^{-1}(U)$$

finalmente resta notar que las contenciones anteriores se mantienen si componemos con $f_1$ esto es; $f_1f_2(K) \subset U,$ de esta manera $f_1f_2 \in \mathcal{U}(K,U)(gh)$, es decir la función $\circ$ es continua. Concluimos que $(Hom(S^1), \tau_{CA})$ es un grupo topológico. Hablaremos ahora de un término importante, el soporte de un homeomorfismo, veremos unas propiedades sencillas que nos permitiran ahorrar tiempo en las explicaciones importantes. 


 
  En el ejemplo \ref{ej:Cir_un_coc} hablamos de un espacio cociente que es homemorfo a la circunferencia y vimos que el grupo de homeomorfismos de la circunferencia es un grupo topológico, estudiaremos ahora que el grupo es conexo. Consideremos a $S^1$ como el cociente $\mathbb{R} / \mathbb{Z}$ como en el ejemplo el ejemplo \ref{ej:Cir_un_coc} y sean $\tilde{f},$ $\tilde{g} \in Hom(S^1)$ existen $f,$ $g: \mathbb{R} \to \mathbb{R}$ homeomorfismos monótonos y crecientes tales que $\pi(f)=\tilde{f}$ y $\pi(g)=\tilde{g}$. Consideremos a,

\begin{align*}
H(x,t)=tg(x)+(1-t)f(x),
\end{align*}

notemos que para un intervalo entero y un punto en él, se tiene que, 

\begin{align*}
H(x+1,t) & = tg(x+1)+(1-t)f(x+1)= tg(x)+t+(1-t)f(x)+(1-t) \\
& = tg(x+1)+(1-t)f(x+1) + t +(1-t)= H(x,t)+1.
\end{align*} 

Como $f$ y $g$ homeomorfismos  monótonos y crecientes dados $t_0 >0$ fijo, $x$, $y \in S^1$ tales que $x <y$ se tiene que, 

\begin{align*}
H(x,t_0) & = t_0g(x)+(1-t_0)f(x) < t_0g(y)+(1-t_0)f(y)=H(y,t_0),
\end{align*} 

tenemos $H(x,t)$ también es monótono y por tanto $H(x,t)$ es una función biyectiva más aún la función inversa tambien es creciente. Al ser una combinación lineal de funciones es continua, tenemos que $\pi(H(x,t))$ es un homeomorfismo en $S^1$ y $\pi(H(x,t))$ es una homotopía entre $g$ y $f$ concluimos que $\Hom(S^1)$ es un grupo topológico conexo. 

\addcontentsline{toc}{section}{Simplicidad del grupo $\Hom(S^1)$}
\section*{Simplicidad del grupo $\Hom(S^1)$}
Estamos en condiciones para demostrar la simplicidad del grupo de homeomorfismo del circulo. El siguiente lema nos será útil en la demostración de la simplicidad del grupo de homeomorfismos de la circunferencia que preservan orientación. 


\begin{lm}
Sea $N$ subgrupo normal de $Hom(S^1)$. Dado $n \in N$ un homemorfismo distinto de la identidad entonces existe $J$ un subintervalo de $S^1$ tal que  $
n(J) \cap J = \emptyset.$
\end{lm}

\begin{proof}
Sea $n:S^1 \to S^1$ homeomorfismo distinto de la función identidad por tanto existe $x \in S^1$ tal que $n(x) \neq x$.  Por la continuidad de $n$ existe $U$  abierto de $S^1$ tal que $n|_U$ no es la identidad, sin pérdida de generalidad podemos suponer que $U$ es un intervalo abierto de $S^1$. Por otro lado, como el espacio $S^1$ es espacio métrico y compacto se tiene que es normal, sin pérdida de generalidad existe un intervalo abierto $V$ vecindad de $n(x)$ tal que $\overline{V} \cap \overline{U} = \emptyset. $ 

Como $Hom(S^1)$ actúa de manera transitiva sobre intervalos de $S^1$ existe un homeomorfismo $h$ tal que $h(U)=V$, en particular se tiene que, $\overline{V}=\overline{h(U)}=h(\overline{U})$ resta notar que, 

\begin{align*}
\emptyset=\overline{U} \cap \overline{V} = \overline{U} \cap \overline{h(U)}=\overline{U} \cap h(\overline{U}).
\end{align*}

Haciendo $J = \overline{U}$ se concluye el resultado. 
\end{proof}

\begin{ob}
Sean $N$ subgrupo normal de un grupo $G$ y $n \in N$. Para cualquier $g \in G$ el elemento $n^{g}$ es por definición un elemento de  $N$ más aún $[n,g]=ngn^{-1}g^{-1}$ también es elemento de $N$.
\end{ob}

\begin{lm}\label{lm:homeo_separador_de_intervalos}
Sea $N$ subgrupo normal  de $Hom(S^1)$. Dado $f \in Hom(S^1)$ soportado en un subintervalo $I$ de $S^1$ existe $n \in N$ tal que $n(I) \cap I = \emptyset.$
\end{lm}

\begin{proof}
Sean  $n_0 \in N$ distinto de la identidad, del lema anterior existe un intervalo $I$ tal que $n_0(I) \cap I =\emptyset.$ Como $\Hom(S^1)$ actúa transitivamente en los subintervalos de  $S^1$ existe $h$ en $\Hom(S^1)$ tal que $h(I) =J$, como $h$ es un homeomorfismo al tomar la función inversa de $h$ se obtiene que $I=h^{-1}(J)$, sustituyendo esto en la parte previa se obtiene que,

$$n_0(h^{-1}(J)) \cap I= \emptyset,$$

del hecho de que $h$ es biyectiva, tomando ahora la imagen directa bajo $h$ se tiene que,

$$h(n_0(h^{-1}(J)) \cap I)=h(n_0(h^{-1}(J))) \cap h(I)= \emptyset,$$

es decir $h(n_0(h^{-1}(J))) \cap J= \emptyset.$ Consideremos a $n=n_0^{h}$, tenemos que $n$ es un homeomorfismo que  cumple la propiedad deseada y $n \in N$.

%, además del lema \ref{lm:obs_a} tenemos que $f=g$ en $I$ y  $f=h^{-1}f^{-1}h$ en $h^{-1}(I)$.
\end{proof}

En resumen, dado un homeomorfismo soportado un sub intervalo $I$, podemos encontrar otro homeomorfismo tal que separa los puntos de $I$. La técnica de considerar conjugados es muy usada en los artículos consultados y en este texto.

\begin{ob}
Sean $f_1$ y $f_2 \in Hom(S^1)$  soportados en $I$, un sub intervalo de $S^1$, de la demostración del lema \ref{lm:homeo_separador_de_intervalos}
también existen $n_1$, $n_2 \in N$ tal que $I$, $n_2(I)$ y $n_2(I)$ son ajenos. Notemos del lema anterior  para $f_1$ existe $n_1$ que cumple el resultado y que $I \cup n_1(I)$ es un conjunto de soporte para $f_2$ del lema anterior tenemos que existe $n_2 \in N$ tal que

\begin{align*}
n_2(n_1(I) \cup I)\cap ( n_1(I) \cup I)= \emptyset
\end{align*}

notemos que 

\begin{align*}
n_2( I)\cap I \subset n_2(n_1(I) \cup I)\cap ( n_1(I) \cup I),
\end{align*}
y
\begin{align*}
n_2( I)\cap n_1(I) \subset n_2(n_1(I) \cup I)\cap ( n_1(I) \cup I).
\end{align*}

\end{ob}

  



%Lema dos para la simplicidad del círculo. 

\begin{lm}\label{lm:dos}
Sea $N$ un subgrupo normal de  $\Hom(S^1)$. Se tiene que el conmutador de dos elementos de  $\Hom(S^1)$ soportados en un mismo intervalo es un elemento de $N$.
\end{lm}

\begin{proof}	
Sean $f_1$ y $f_2 \in \Hom(S^1)$  soportados en $I$, un sub intervalo de $S^1$, luego por el lema \ref{lm:homeo_separador_de_intervalos} existen  $n_1$, $n_2 \in N$ tales que $I,$ $n_1^{-1}(I)$ y $n_2^{-1}(I)$ son conjuntos disjuntos. Definimos a  $g_1=[n_1^{-1},f_1^{-1}]$ y a $g_2=[h_2^{-1},f_2^{-1}]$ que por  normalidad  son elementos de  $N$. Además por el lema \ref{lm:obs_a} se tiene que,

    \begin{enumerate}
        \item $g_1=f_1$  y $g_2=f_2$ en  $I$, 
        \item $g_1$ está soportando en $n_1(I) \cup I$ y que 
        \item  $g_2$ está soportando en $n_2(I) \cup I$.
    \end{enumerate} 

Como $I$, $n_1(I)$ y $n_2(I)$ son conjuntos ajenos, dado $x \in n_1(I)$ se cumple que, 

$$[g_1,g_2](x)=[(f_1^{-1})^{[n_1^{-1}]},\Ide_{S^1}](x)=\Ide_{S^1}(x)$$

y $$[f_1,f_2](x)=\Ide_{S^1}(x).$$

El caso es análogo si $x \in n_2(I)$, por tanto al estudiar los puntos del conjunto de soporte  en cada caso concluimos que,
$$[g_1,g_2]=[f_1,f_2].$$
\end{proof}

Vamos a usar una observación que previamente argumentamos.

\begin{ob} 
De la proposición \ref{pr:Derivados_y_generadores} si un grupo $G$ es generado por un subconjunto $X$, su primer grupo derivado $G'$ es generado por conjugados de conmutadores de elementos de $X$.
\end{ob}

Este lema es una parte de Theorem 4.3 que se puede consultar en \cite{ander}.

\begin{lm}\label{lm:tres}
Sea $N$ un subgrupo normal de  $Hom(S^1)$. Dados  $I_1$, $I_2$ e $I_3$ intervalos que cubren a $S_1$ que cumplen,

\begin{enumerate}
	\item   $I_1 \cap I_2 \cap I_3=\emptyset$,
	\item pero no vacía dos a dos.
\end{enumerate} 
Denotemos a $I_{12}=I_1 \cap I_2$, $I_{23}=I_2 \cap I_3$ y  $I_{31}=I_3 \cap I_1$   y consideremos a 

 $$G_j=\langle \{g \in Hom(S^1): g \text{ está soportando en } I_{ji}\} \rangle$$ 
donde $j=1,2, 3$, los subgrupos generados por los elementos que tienen soporte $I_{ji}$ respectivamente. Se cumple para $G= \langle G_1 \cup G_2 \cup G_3 \rangle$ que $$G' \subset N.$$
\end{lm}

\begin{proof}
Para ello, notemos de la observación anterior $G'$ es generado por los conjugados de los conmutadores de $G_1\cup G_2 \cup G_3$. Sea $g \in G'$ del lema \ref{lm:obs_a} tenemos que 

\begin{align*}
g=\prod_i^n[h_i^{k_{2_i}},f_i^{k_{1_i}}]
\end{align*}

donde $h_i$ y $f_i \in G_1 \cup G_2 \cup G_3$ y $k_{1_i}$ y $k_{2_i} \in Hom(S^1)$ notemos que, tanto $h_i$ como $f_i$ tienen soporte contenido en un mismo intervalo, del lema \ref{lm:dos} obtenemos que, $$[h_i^{k_{2_i}},f_i^{k_{1_i}}] \in N,$$ para $i=1, \cdots n$.  Por tanto $g \in N$ y en consecuencia $G'  \subset N.$
\end{proof}

El siguiente teorema es presentado en \cite{ander} como Theorem 4.3.

\begin{te}
El grupo $\Hom^+(S^1)$ de homeomorfismos que preservan la orientación es simple.
\end{te}

\begin{proof}
Sean $N$ un subgrupo normal de  $\Hom^+(S^1)$ y  $I_1$, $I_2$ e $I_3$ intervalos que cubren a $S_1$ que cumplen las propiedades del lema \ref{lm:tres}. Sean  $K_1 \subset I_{12}$, $K_2 \subset I_{23} $ y $K_3 \subset I_{31}$ conjuntos compactos no vacíos, la unión

$$K= \cup_j K_i,$$

 es un conjunto compacto. Considerando al abierto,
 
 $$U=\cup_j \Int(I_{ji}),$$
 
 tenemos al elemento básico $\mathcal{U}(K,U)$ y claramente $Id_{S^1} \in  \mathcal{U}(K,U)$ así este conjunto básico es no vacío. Por la equivalencia entre la topología compacto-abierta y la de convergencia uniforme existe $\varepsilon >0 $ tal que en métrica uniforme se tiene que $B_\varepsilon(Id) \subset \mathcal{U}(K,U)$.

Sean $f \in B_\varepsilon(Id) $, $x$, $y$ y $z$ en $I_{12}$, $I_{23} $ y $I_{31}$ respectivamente, notemos que $f(x)$, $f(y)$ y $f(z)$ están en los interiores de las intersecciones $I_{ji}$ respectivamente, además existen $g_1$, $g_2$ y $g_3 \in \Hom(S^1)$ tal que tienen soporte en $I_{12}$, $I_{23} $ y $I_{31}$ respectivamente y son $f$ en vecindades de $x$, $y$ y $z.$ Más aún se tiene que 


\begin{align*}
g_3^{-1}g_2^{-1}g_1^{-1}f \equiv Id \in G,
\end{align*}

%por tanto existen $h_i \in G_i$ tales que

%\begin{align*}
%f=g_1g_2g_3 h_1h_2h_3= g_1h_1g_2 h_2g_3h_3
%\end{align*}

por tanto $B_\varepsilon(Id) \subset G$. Como $Hom_+(S ^{1})$ es un grupo conexo tenemos que $\langle B_\varepsilon(Id) \rangle= \Hom^+(S^1)$ de esta manera tenemos que $\Hom^+ (S^1)= G.$

 \end{proof} 

 
 En \cite{ander} en proposition 5.11, se demuestra que el grupo de homeomorfismos $Hom(S^1)$ es perfecto, es decir que coincide con su primer grupo conmutador por tanto $Hom(S^1)'=G'$. Del lema \ref{lm:dos} para todo $N$ subgrupo normal de $\Hom(S^1)$ se tiene que $Hom(S^1)' \subset N$ tenemos así que el grupo $Hom(S^1)'$ es simple.  
 
 
 \begin{cn}
 Estudiar la demostración de proposición 5.11
 en  \cite{ander} está fuera de nuestros fines, requiere agregar en síntesis la sección 5 de \cite{ander} que nos habla de la dinámica del grupo $Hom(S^1)$ sobre $S^1$, es un tema rico en matemática como proposition 5.9 como también en técnicas sobre el uso de conmutadores, pero haría de nuestra sección un texto largo para concluir el teorema anterior, en este texto estamos estudiando la simplicidad de grupos.
\end{cn}



%%%%%%%%%%%%   Conjunto de Cantor
\addcontentsline{toc}{chapter}{La simplicidad de grupos de homeomorfismos}
\chapter*{La simplicidad de grupos de homeomorfismos}
Vamos a usar los resultados de \cite{ander} para demostrar la simplicidad de los grupos de homeomorfismos del conjunto de Cantor y de la curva de Sierpiński. Estos espacios son interesantes pues comparten propiedad muy similares. 

\addcontentsline{toc}{section}{Conjunto de Cantor}
\section*{Conjunto de Cantor}
El conjunto de Cantor, un ejemplo muy importante para el análisis metemático, es un espacio topológico que mediante su topología es posible clasificar otros espacios con ciertas  propiedades iguales a las de él. Vamos a usar un resultado para la construcción del conjunto de Cantor el siguiente teorema puede encontrarse en \cite{top_willd} como theorem 17.4.

\begin{te}
La intersección anidada de conjuntos compactos anidados es no vacía. 
\end{te}
Además de el siguiente resultado que se encuentra en \cite{top_prieto} capítulo VIII como Teorema 1.18.
 
 \begin{te}
 Sea $X$ un espacio Hausdorff, toda intersección no vacía de conjuntos compactos es un conjunto compacto. 
 \end{te}
 
 Por tanto, de los resultados anteriores tenemos que en un espacio compacto y Hausdorff la inteserción anidada de conjuntos compactos es compacta. 
 
\subsection*{Construcción del conjunto de Cantor ternario}
Vamos a definir conjuntos de manera recursiva, para el primer paso  sean $I=[0,1]$ y $J_1=(1/3,2/3)$. Definimos los conjuntos

\begin{enumerate}
\item $F_{11} = [0,1/3]$ y $F_{12} = [2/3,1]$. Notemos que la longitud de estos intervalos es 1/3.
\item $C_1 = I \setminus J_1= F_{11} \cup F_{12}$. 
\end{enumerate} 

En el primer paso, al intervalo unitario $I$ lo dividimos en tres y  le hemos quitado el intervalo de en medio identificado como $J_1$.
Para el segundo paso, para los intervalos $F_{11}$ y $F_{12}$ vamos a dividirlos en tres subintervalos, los intervalos $J(F_{11})=(1/3^2,2/3^2)$ y $J(F_{12})=(7/3^2,8/3^2)$. Definimos a los intervalos

\begin{enumerate}
\item $J_2$=$J(F_{11}) \cup J(F_{12})$,
\item $F_{21}=[0,1/3^2]$,  $F_{22}=[2/3^2,3/3^2]$, $F_{23}=[6/3^2,7/3^2]$ y $F_{24}=[8/3^2,1]$, notemos que la longitud de estos intervalos es $1/3^2$.
\item $C_2=I \setminus J_2$. 
\end{enumerate}

En el segundo paso, para los dos intervalos restantes del paso anterior, $F_{11}$ y $F_{12}$, dividimos cada intervalo en tres y retiramos el intervalo de en medio es decir, a $F_{12}$ le quitamos el intervalo $J(F_{11})$ obteniendo ahora dos intervalos $F_{21}$ y $F_{22}$. Para el intervalo $F_{22}$ le hemos quitado el intervalo $J(F_{22})$ de este modo tenemos dos intervalos $F_{23}$ y $F_{24}$. La notación representa lo siguiente, $F_{ij}$ se define como la $j$-ésima componente restante obtenida en el paso $i$. 

Repitiendo de manera infinita, obtenemos sucesiones de conjuntos $(J_n)_{n=1}^\infty$ y $(C_n)_{n=1}^\infty$. La familia de intervalos $J_n$ es la unión disjunta de $2^{n-1}$ intervalos abiertos y $C_n$ es una sucesión decreciente de intervalos cerrados, donde cada $C_n$ es unión disjunta de $2^n$ intervalos cerrados. La longitud de cada $J_n$ y $C_n$ es $1/3^n.$

Consideremos finalmente a $J=\bigcup_{n }^{\infty}  J_n$ definimos al \textbf{conjunto Cantor ternario} como $$C=\bigcap_{n=1}^\infty C_n=I \setminus J.$$ 

En lo siguiente veremos que el conjunto de Cantor es un espacio topológico con una propiedad de universal. 

\subsection*{Topología del conjunto de Cantor}

El intervalo $I$ es un conjunto compacto por el teorema de Hein-Borel, \cite{cal_Paez} Teorema 2.6. Cada conjunto $F_{ij}$ es un subintervalo cerrado y acotado de $I$ por tanto es compacto. Además cada $C_n$ es una unión finita de conjuntos compactos en consecuencia cada conjunto $C_n$ es un conjunto compacto para cada $n \in \mathbb{N}$. Por construcción se cumple que, dados $n < m $ índices, $C_m \subset C_n$. Tenemos por tanto que los $C_n$ forman una familia de conjuntos compactos, no vacíos y anidados de esta manera el conjunto $C$ es un conjunto compacto. 

 El conjunto de Cantor también cumple ser totalmente disconexo, \cite{top_willd} example 26.13.b. En otras palabras, para cada $x \in C$ se tiene que $Q(x)=\{x\}$. El siguiente teorema es importante muy importante y una demostración se encuentra en \cite{top_willd} en la página 216.

\begin{te}\label{te:Cantor_universal}
Salvo homeomorfismo el conjunto de Cantor es el único espacio métrico, totalmente disconexo, compacto y perfecto. 
\end{te}


Con el trabajo de Anderson en \cite{ander} el grupo de homeomorfismos del Cantor es simple, vamos a estudiar sus resultados que nos darán la simplicidad de grupos.  

\addcontentsline{toc}{section}{La simplicidad del grupo de Homeomorfismos del conjunto de Cantor}
\section*{La simplicidad de grupos de Homeomorfismos}

A partir de ahora a menos que se indique lo contrario vamos a añadir la hipótesis de que $X$ es un espacio Hausdorff además $\Hom^0(X)$ denotará a la familia de los homeomorfismos $h:X \to X$ tal que existe $U$ abierto en $X$ de manera que $h|_U=\Ide_U.$ Vamos a introducir una de las nociones importantes y fundamentales del trabajo de Anderson. En la siguiente definición un conjunto es considerado degenerado si consiste en un conjunto con un solo elemento.

\begin{df}
Sean $X$ espacio topológico. Denotamos a $2^X$ la colección de los subconjuntos cerrados y no vacíos de $X$.  Sea $\Kst \subset 2^X$ tal que;
 	
	\begin{enumerate}
	\item los elementos de $\Kst$ son no degenerados y homeomorfos unos con otros,
	\item para cada $U$ abierto de $X$ existe $K \in \Kst$ tal que $K \subset U,$
	\item para $K \in \Kst$, $\Cla(K^c) \in \Kst.$
	\end{enumerate}

Diremos que $\Kst$ es una \textbf{estructura de rotación} para $X$ o que $X$ tiene una $\Kst$-\textbf{estructura de rotación}.
\end{df}

Vamos a recurrir a un teorema sobre dimensión y sistemas de vecindades. Para ello puede consultarse a \cite{top_willd} theorem  29.7 y example 29.8.

\begin{te}
El conjuto de Cantor tiene un sitema de vecindades de abiertos y cerrados. 
\end{te}

Vamos a usar esta propiedad del conjunto de Cantor, existe una base $\Kst$ de conjuntos abiertos y cerrados, notemos que las propiedades 2 y 3 de la definición se satisfacen por ser base para una topología. Resta ver la primera parte y tan solo el hecho de que son homeomorfos. Dados $K$ y $W$ conjuntos en $\Kst$ notemos que cada conjunto es cerrado y por tanto son compactos, se hereda la total disconexión y perfección como subespacios también son métricos, por el teorema \ref{te:Cantor_universal} existe $f:K \to W$ homeomorfismo.

Notemos que los argumentos son aplicables a $\mathcal{C} \setminus K$ y  $\mathcal{C} \setminus W$, por tanto existe un homeomorfismo $h:\mathcal{C} \setminus K \to \mathcal{C} \setminus W$ del lema del pegado de funciones tenemos que existe un homeomorfismo $f \cup h: \mathcal{C} \to \mathcal{C}$ y que  manda a $K$ en $W.$  Se tiene que el conjunto de Cantor tiene una $\Kst$ estructura de rotación. 


\begin{ob}\label{ob:hom_sop_en_Ks}
Sea $g \in \Hom^0(X)$ distinto de la identidad entonces existe un elemento $K \in \Kst$ de manera que $g$ está soportado en $K$. Para esto, sea $g \in \Hom^0(X)$ existe un abierto $U$ en $X$ tal que $g|_U=\Ide_U$, por la definición de $\Kst$-estructura existe un conjunto $K_1 \in \Kst$ tal que $K_1 \subset U$ y tomando complementos de esto último se tiene que 

	\begin{align*}
	\Sop(g) \subset X \setminus U \subset X \setminus K_1
	\end{align*}

al tomar la cerradura de cada conjunto tenemos que,

	\begin{align*}
	X \setminus U \subset \overline{ X \setminus K_1}
	\end{align*}

puesto que $X \setminus U$ es cerrado, por la tercera condición de $\Kst$ estructura se sigue que $\overline{ X \setminus K_1} \in \Kst$ y por tanto $g$ está soportado en un elemento $K = \overline{ X \setminus K_1}$ de $\Kst$.
\end{ob}



\begin{lm}\label{lm:K_separado_por_h}
Sea $h \in \Hom(X)$ distinto de la identidad, entonces existe $K \in \Kst$ tal que $h(K) \cap K = \emptyset.$
\end{lm}

\begin{proof}
Como $h$ es distinto de la identidad, existe $x \in X$ tal que $h(x) \neq x$. Más aún, como $X$ es un espacio Hausdorff sin pédida de generalidad existen $U$ y $V$ vecindades de $x$ e $y$ respectivamente tales que $U \cap V  = \emptyset,$ por la continuidad de $h$ se cumple que $h(U) \subset V$, de la definición de $\Kst$ estructura, existe $K$ tal que $K \subset U$ y por tanto $h(K) \subset V$ tenemos así que, $K \cap h(K) = \emptyset.$

\end{proof}

\begin{ob}
Del lema anterior notemos que aplicando la función $h^{-1}$, la inversa de la función $h$, tenemos que $K \cap h^{-1}(K) = \emptyset,$ teniendo así que,

	\begin{align*}
    K \cap (h^{-1}(K) \cup h(K)) = \emptyset.
	\end{align*}
	
\end{ob}


\begin{df}\label{df:Suc_rot}
Sea $X$ un espacio con $\Kst$-estructura de rotación. Si existe una sucesión de conjuntos $(K_i)_{i\in \mathbb{Z}} \subset \Kst$ tal que,

	\begin{enumerate}
	\item $(K_i)_{i\in \mathbb{Z}}$ es una sucesión de conjuntos disjuntos. 
	\item Existe $U$ abierto en $X$ tal que $\bigcup_{i\in \mathbb{Z}} K_i \subset U$.
	\item $\Cla(\cup \Kst)- \cup \Kst=\{p \}$ con $p \in X$ y es tal que para cualquier vecindad $U$ de $p$ contiene todas excepto una cantidad finita de elementos de $\Kst.$
	\end{enumerate}

Decimos que $X$ tiene una \textbf{sucesión de rotación} y a $(K_i)_{i\in \mathbb{Z}}$ le diremos una \textbf{sucesión de rotación} en $X$.
\end{df}


Sean $\mathcal{C}$ un conjunto de Cantor y $\beta$ una base de abiertos y cerrados para $\mathcal{C}$. Definiremos a una familia de conjuntos, 
	
	\begin{enumerate}
		\item Sea $F_{24}$ de la construcción del Cantor, como $\Kst$ es base existe $K_0 \subset \mathcal{C} \cap F_{24}$.
		
		\item  Consideremos también los conjuntos $F_{22}$ y $F_{23}$ nuevamente, como $\Kst$ es base existen $K_1$ y $K_{-1}$ tales que $K_1 \subset \mathcal{C} \cap F_{22}$ y $K_{-1} \subset \mathcal{C} \cap F_{23}$.
		
\item Por este proceso existen $K_{i}$ y $K_{-i} $ conjuntos abiertos y cerrados tales que $K_i \subset F_{i2}$ y $K_{-i} \subset F_{i3}$.
	\end{enumerate}
Sea $p=0$, es claro que $p \in \mathcal{C}$, la familia de conjuntos $(K_i)_{i \in \mathbb{Z}}$ es una sucesión de rotación para el conjunto de Cantor. En efecto se cumple las primeras dos condiciones de la definición por la  construcción de los conjuntos $K_i$, note además que los conjuntos son abiertos y cerrados. Resta ver $0 \in \Cla(\bigcup_i K_i)$. 

Sea $r >0 $ y consideremos a $B_r(0) \subset \mathcal{C}$. Como la sucesión $1/3^n \to 0$ cuando $n \to \infty$ existe $N \in \mathbb{N}$ tal que $1/3^n < r$ para $n \geq N$, por tanto existen $F_{n2}$ y $F_{n3}$ tales que los extremos derechos de estos intervalos son menores que $r$ y en consecuencia $K_{n}$ y $K_{-n}$ están contenidos en $B_r(0)$ para $n \geq N.$ De esta manera se tiene que $\Cla(\cup \Kst)- \cup \Kst=\{0\}$ tenemos que $\mathcal{C}$ tiene una sucesión de rotación. 


Recordemos la notación, dado $G$ subgrupo de $\Hom$, $G^0$ es una familia de homeomorfismos para los cuales existe un abierto donde son la identidad.

\begin{df}
Sean $X$ un espacio con una $\Kst$-estructura y $G$ un subgrupo de $\Hom(X)$. Decimos que $X$ tiene \textbf{rotación}-$(G,\Kst)$ si para cualquier $K \in \Kst$ existe una sucesión de rotación $(K_i)_{i \in \Ent}$ tal que 

	\begin{enumerate}
	\item $\cup_{i \in \Ent} K_i \subset K$ y
	\item existen $h_1$, $h_2 \in G^0$ con soporte en $K$ tales que;  

		\begin{enumerate}
		\item $h_1(K_i)=K_{i+1}$ para cada $i$.
		
		\item  \begin{itemize}
					\item $h_2|_{K_0}=h_1|_{K_0}$, 
					\item $h_2|_{K_{2i}}=(h_1^2)^{-1}|_{K_{2i}}$,
					\item para toda $i >0$ y $h_2|_{K_{2i-1}}=h_1^2|_{K_{2i-1}}$ para toda $i >0$.
			   \end{itemize}		 
			
		\item Si para cada $i$, $f_i \in \Hom^0(X)$ soportado en $K_i$, entonces existe $f \in \Hom^0(X)$ soportado en $\cup K_i$ tal que $f|_{K_i}=f_i|_{K_i}$ para cada $i$.
		
		\item Para cualquier $K' \in \Kst$ existe $\varphi \in \Hom(X)$ tal que $\varphi(K')=K.$

		\end{enumerate}

	\end{enumerate}
\end{df}

Usaremos en muchas ocasiones el lema \ref{lm:obs_A} que vimos anteriormente y es importante tener en cuenta la observación \ref{ob:efecto_soporte}. 

\begin{lm}\label{lm:obs_A_2}
Sean $K \subset X$ y $g \in \Hom^0(X)$ soportado en $K$. Para cualquier $h \in \Hom(X)$ se tiene que,
 
	\begin{enumerate}
	\item  $g^{h^{-1}}$ está soportado en $h^{-1}(K)$.
	\item Si $h^{-1}(K) \cap K = \emptyset$ y $h^{-1}(K) \cup K \neq X$  entonces
		\begin{enumerate}
		\item $[h^{-1},g^{-1}]$ está soportado en $h^{-1}(K) \cup K$,
		\item $[h^{-1},g^{-1}]|_K=g|_K$ y 
		\item $[h^{-1},g^{-1}]|_{h^{-1}(K)}=g^{h^{-1}}|_{h^{-1}(K)}$
		\end{enumerate}	
	\end{enumerate}
\end{lm}

y la observación siguiente

\begin{ob}\label{ob:efecto_soporte_2}
La composición de funciones con soportes ajenos. Sean $f$, $g$ dos homemorfismos tales que $K_f$ y $K_g$ son conjuntos de soporte respectivamente y tales que  $K_f \cap K_g = \emptyset$ y $K_f \cup K_g \neq X$ entonces para $h=fg$ se cumple que, $h=g$ en  $ K_g$, $h=K_f$ en $K_f$ y $h$ tiene soporte en $K_f \cup K_g$.
\end{ob}

\begin{nt}
Para el siguiente resultado es importante mencionar que, cuando nos referimos que $g$ es el producto de cuatro conjugados de $h$ y $h^{-1}$ esto se tiene salvo el orden en los factores. 
\end{nt}

\begin{lm}\label{lm:lema_1}
Sea $X$ un espacio con rotación $(G,\Kst)$ y $h \in \Hom(X)$ no trivial, existe $K_0 \in \Kst$ tal que para todo $g \in G_0$ soportado en $K_0$ se tiene que $g$ es el producto de cuatro conjugados de $h$ y $h^{-1}$. 
\end{lm}



\begin{proof}
Sea $h \in \Hom(X)$ no trivial y $K \in \Kst$ como en el lema \ref{lm:K_separado_por_h}. Como $X$ tiene rotación- $(G, \Kst)$ existen una sucesión de rotación $(K_i)_{i \in \Ent}$, con $\cup_{i \in \mathbb{Z}} K_i \subset K$ y existen homeomorfismos $\phi_1$, $\phi_2$ soportados en $K$. Afirmamos que el conjunto $K_0$ de la sucesión es el conjunto que cumple con el lema, sea $g_0$ un homeomorfismo soportado en $K_0$. Vamos a construir un homeormorfismo auxiliar $w$. Consideremos los homeomorfismos, 

\begin{align*}
f_i= \begin{cases} 
g^{\phi_1^i} & \text{ para } i \geq 0 \\
e & \text{ para } i <0. 
\end{cases}
\end{align*}

Del lema \ref{lm:obs_A_2}, $f_i$ está soportado en $\phi_1^i(K_0)=K_i$, de la definición de rotacionalidad, existe un homeomorfismo $f$ tal que

\begin{enumerate}
	\item $f$ está soportado en $\bigcup_i K_i$,
	\item $f |_{K_i}=f_i|_{K_i},$
\end{enumerate}

Sea $\tilde{f}=[h^{-1},f^{-1}]$,  por el lema \ref{lm:obs_A_2} tiene soporte en el conjunto

\begin{align*}
 Y= \left(\bigcup_i h^{-1}(K_i) \right)\cup \left(\bigcup_i K_i \right).
\end{align*}

Por otro lado definamos a $\tilde{h}=\phi_2^{h^{-1}}\phi_1^{-1}$ por el lema \ref{lm:obs_A_2} tenemos que $\phi_2^{h^{-1}}$ tiene soporte en $h^{-1}(K)$ notemos además que $\phi_1^{-1}$ tiene soporte en $K$. De la observación \ref{ob:efecto_soporte} tenemos que $\tilde{h}$ tiene soporte en el conjunto $Y.$ Finalmente vamos a definir a $w = [\tilde{h}^{-1},\tilde{f}^{-1}]$ notemos que,

\begin{align*}
w &  =\tilde{h}^{-1}\tilde{f}^{-1}\tilde{h}\tilde{f} =\tilde{h}^{-1}([h^{-1},f^{-1}])^{-1}\tilde{h}([h^{-1},f^{-1}]) \\
& = \tilde{h}^{-1} (f^{-1}h^{-1}fh)^{-1}\tilde{h}(h^{-1}f^{-1}hf)\\
& = (\tilde{h}^{-1} f^{-1}h^{-1}f \tilde{h})(\tilde{h}^{-1} h\tilde{h})(h^{-1})(f^{-1}hf)\\
& = (h^{-1})^{(\tilde{h}^{-1} f^{-1})}h^{(\tilde{h}^{-1})}(h^{-1})^{(\Ide)}h^{(f^{-1})}
\end{align*}


Notemos que $w$ es el producto de cuatro conjugados de $h$ y $h^{-1}$ en orden alterno. Para terminar  es suficiente demostrar que $w = g$. Veremos la igualdad en el conjunto $\cup K_i$, para $\tilde{f}$ del lema \ref{lm:obs_A_2} tenemos que  

\begin{align*}
\tilde{f}|_{\cup_i K_i} = f|_{\cup_i K_i}
\end{align*}

por tanto se tiene que $\tilde{f}|_{K_i} = f_i|_{K_i}$ para $i \geq 0$ y  para $\tilde{h}$ de la observación \ref{ob:efecto_soporte_2} tenemos que, $\tilde{h}|_{K}=\phi_1^{-1}|_{K}$ y por tanto, para cada $K_i$ se tiene que, $\tilde{h}|_{ K_i}=\phi_1^{-1}|_{ K_i}$, dado que los conjuntos $K_i$ son ajenos tenemos entonces que

\begin{align*}
\tilde{h}|_{\cup_i K_i}=\phi_1^{-1}|_{\cup_i K_i}
\end{align*}

de la observación \ref{ob:efecto_soporte_2} tenemos que 


\begin{align*}
w|_{\cup_i K_i} =\tilde{h}^{-1}\tilde{f}^{-1}\tilde{h}\tilde{f}|_{\cup_i K_i} = \phi_1f^{-1}\phi_1^{-1}f|_{\cup_i K_i}.
\end{align*}

En particular para $K_0$, se tiene que $f|_{K_0}=f_0|_{K_0}=g|_{K_0}$ pero por definición de rotación-$(G,\Kst)$ $\phi_1(K_0)=K_1$ y de la observación \ref{ob:efecto_soporte_2} tenemos que $\phi_1g^{-1}\phi_1^{-1}(K_0)=\phi_1 \Ide g|_{K_0} =\Ide|_{K_0}$. De esta manera tenemos que $w|_{K_0}=g_0|_{K_0}.$ Más aún, para $i>0$ se tiene que, $w|_{K_i} = \phi_1f^{-1}\phi_1^{-1}f|_{ K_i}$, por construcción tenemos que $f|_{K_i}=f_i|_{K_{i}}$ por tanto

\begin{align*}
w(K_i) & = \phi_1f^{-1}\phi_1^{-1}f_i(K_i)=\phi_1f^{-1}\phi_1^{-1}(g^{\phi_1^i})(K_i)\\
& = \phi_1f^{-1}\phi_1^{-1}(K_0)\\
& = \phi_1f^{-1}(K_{i-1})
\end{align*} 

notemos que $f|_{K_{i-1}}=f_{i-1}|_{K_{i-1}}$, de esto se sigue que, 

\begin{align*}
w(K_i) &  = \phi_1f_{i-1}^{-1}(K_{i-1})=K_i
\end{align*} 
tenemos por la observación \ref{ob:efecto_soporte_2} que, 

\begin{align*}
w|_{K_i} & = \phi_1f^{-1}\phi_1^{-1}f|_{K_i}=\phi_1f_{i-1}^{-1}\phi_1^{-1}f_i|_{K_i}\\
& = \phi_1 (g^{-1})^{\phi_1^{i-1}} \phi_1^{-1} (g^{\phi_1^i})|_{K_i}\\
& = (g^{-1})^{\phi_1^i} (g^{\phi_1^i})|_{K_i}= \Ide|_{K_i}
\end{align*}


tenemos que $w|_K=g|_K.$ Veamos ahora en el conjunto $h^{-1}(K)$, del lema \ref{lm:obs_A_2}

\begin{align*}
\tilde{f}|_{\bigcup_i h^{-1}(K_i)} = (f^{-1})^{h^{-1}}|_{\bigcup_i h^{-1}(K_i)}.
\end{align*}

y por la observación \ref{ob:efecto_soporte_2}  para $\tilde{h}$ tenemos que,

\begin{align*}
\tilde{h}|_{\bigcup_i h^{-1}(K_i)}=\phi_2^{h^{-1}}|_{\bigcup_i h^{-1}(K_i)}
\end{align*}

tenemos entonces que $w|_{\bigcup_i h^{-1}(K_i)}= (\phi_2^{-1})^{h^{-1}}(f)^{h^{-1}}\phi_2^{h^{-1}}(f^{-1})^{h^{-1}}|_{\bigcup_i h^{-1}(K_i)}$, pero 

\begin{align*}
(\phi_2^{-1})^{h^{-1}}(f)^{h^{-1}}\phi_2^{h^{-1}}(f^{-1})^{h^{-1}}=(\phi_2^{-1} f \phi_2 f^{-1})^{h^{-1}}
\end{align*}

entonces resta ver que ocurre con  $\phi_2^{-1} f \phi_2 f^{-1}$ en $h^{-1}(K)$. De lo anterior $f^{-1}|_{K_0}=g_0^{-1}|_{K_0}$ y para $i>0 $ $f^{-1}|_{K_i}=f_i^{-1}|_{K_i}=(g_0^{-1})^{\phi_1^i}|_{K_i}$. Vamos a separar los casos en pares e impares en los índices también es importante recordar la relación entre $\phi_1$ y$\phi_2$ en la definición de rotación. Para $K_{2i}$ tenemos que

\begin{align*}
\phi_2^{-1}f\phi_2f^{-1}|_{K_{2i}} = & \phi_2^{-1}f \phi_1^{-2} f_{2i}^{-1}|_{K_{2i}} = \phi_2^{-1}f_{2i-2} \phi_1^{-2} f_{2i}^{-1}|_{K_{2i}} \\
=& \phi_1^2(\phi_1^{2i-2}g\phi_1^{-2i+2})\phi_1^{-2}(\phi_1^{2i}g^{-1}\phi_1^{-2i})|_{K_{2i}}= \\
= & g^{\phi_1^{2i}}(g^{-1})^{\phi_1^{2i}}|_{K_{2i}}=\Ide|_{K_{2i}}
\end{align*}

y para los conjuntos $K_{2i-1}$ tenemos un proceso similar, 

\begin{align*}
\phi_2^{-1}f\phi_2f^{-1}|_{K_{2i-1}} = & \phi_2^{-1}f \phi_1^{-2} f_{2i-1}^{-1}|_{K_{2i-1}} = \phi_2^{-1}f_{2i-2} \phi_1^{-2} f_{2i}^{-1}|_{K_{2i}} \\
= & \phi_1^{-2}(\phi_1^{2i+1}g \phi_1^{-2i-1})\phi_1^2(\phi_1^{2i-1}g^{-1}\phi_1^{-2i+1})|_{K_{2i-1}} \\
= &  g^{\phi_1^{2i-1}}(g^{-1})^{\phi_1^{2i-1}}|_{K_{2i-1}}\\
= & \Ide|_{K_{2i-1}}.
\end{align*}

y para $K_0$ tenemos que,

\begin{align*}
\phi_2^{-1}f\phi_2f^{-1}|_{K_0} = \phi_1^{-1}(\phi_1 g \phi_1^{-1})\phi_1(g^{-1})|_{K_0}=gg^{-1}|_{K_0}=\Ide|_{K_0}
\end{align*}

\end{proof}

Este resultado se encuentra en \cite{ander} como propiedad (B).

\begin{ob}\label{ob:numero_conjugados}
Cualquier conjugado del producto de conjugados de producto de $h$ y $h^{-1}$ tiene el mismo numero de conjugados de $h$ y $h^{-1}$. Si $f=(g_1 h^{-1} g_1^{-1}) \cdots(g_n h g_n^{-1})$ entonces para cualquier $g$ en $Hom(X)$ se tiene que
\begin{align*}
g f g^{-1} & = g(g_1 h^{-1} g_1^{-1}) \cdots (g_n h g_n^{-1})g^{-1}\\
&= (gg_1 h^{-1} g_1^{-1}g^{-1})\cdots(gg_n h g_n^{-1}g^{-1}).
\end{align*} 
\end{ob}	  
 
Con este lema vamos a demostrar el siguiente resultado, que se encuentra como \cite{ander} Theorem 1.

\begin{te}
Sea $X$ un espacio con rotación $(G,\Kst)$ y $h \in \Hom(X)$ no trivial. Para todo $g \in G^0$ se tiene que $g$ es el producto de cuatro conjugados de $h$ y $h^{-1}$.
\end{te}



\begin{proof}
Sean $g \in Hom_0(X)$ soportado en $K \in \Kst$ y $K_0$ como en las hipótesis del lema, como $X$ tiene rotación-$(G,\Kst)$  existe $\varphi \in G$ tal que 

\begin{align*}
    \varphi(K) = K_0.
\end{align*}

Tomando a $g_0= g^{\varphi}$ por el lema \ref{lm:obs_A_2} se tiene que $g_0$ está soportando en $\varphi(K)=K_0$, del lema  \ref{lm:lema_1} $g_0$ es producto de cuatro conjugados de $h$ y $h^{-1}$,

\begin{align*}
g_0= h^{g_1}(h^{-1})^{g_2}h^{g_3}(h^{-1})^{g_4}
\end{align*}

 entonces
 
\begin{align*}
 g= \left( h^{g_1}(h^{-1})^{g_2}h^{g_3}(h^{-1})^{g_4} \right)^{\varphi^{-1}}
\end{align*}
 de la observación \ref{ob:numero_conjugados} se tiene que $g$ es producto de cuatro conjugados, es decir, 
 
 \begin{align*}
 g= h^{\varphi^{-1}g_1}(h^{-1})^{\varphi^{-1}g_2}h^{\varphi^{-1}g_3}(h^{-1})^{\varphi^{-1}g_4} .
\end{align*}
\end{proof}

\begin{co}
El grupo de homeomorfismos del Cantor es simple. 
\end{co}


\bibliography{biblio}
\bibliographystyle{plain}
\end{document}




%
%\begin{df}
%Sea $X$ un espacio con rotación-$(G,\Kst)$, decimos que $X$ es $(G,\Kst)$-**homogéneo** si
%	
%	
%- para cualquier $K \in \Kst$ y $h \in Hom(X)$ tal que $h(K)\cap K = \emptyset$ entonces existe $K' \in \Kst$ y $\psi \in G_0$, $\psi$ con soporte en $K'$ tal que $\psi|_K=h|_K$.
%- Para cualesquiera $K_1$, $K_2$, $K_3$ y $K_4 \in Im_G(\Kst)$ con $K_1 \cap K_2=K_3 \cap K_4 = \emptyset$, $K_1 \cup K_2 \neq X \neq K_3 \cup K_4$, existe $\varphi \in Hom(X)$ tal que $\varphi(K_1)=K_3$ y $\varphi(K_2) =K_4.$
%


%\begin{cn}
%$Im_G(2^X)$ denotará la colección de todas las imágenes de elementos de $2^X$ bajo los elementos de $G \subset Hom(X)$, esto es, para una famlia de conjuntos $\Kst \subset 2^X$ se tiene que,
%	
%	\begin{align*}
%	Im_G(\Kst)=\{g(K): K \in \Kst  \text{ y } g \in G \}.
%	\end{align*}
%
%\end{cn}	 
%
%
%\addcontentsline{toc}{chapter}{Bibliografía}
%
%\bibliographystyle{plain}
%
%\bibliography{biblio}
%
%\end{document}
%\begin{lm}
%Sea $h \in Hom(X)$ distinto de la identidad, entonces existe $k \in \Kst$ tal que $h(K) \cap K = \emptyset.$
%\end{lm}

%\begin{proof}
%Como $h$ es distinto de la identidad, existe $x \in X$ tal que $h(x) \neq x$. Más aún, como $X$ es un espacio Hausdorff sin pédida de generalidad existen $U(x)$ y $V(h(x))$ tales que 
%
%\begin{align*}
%    U \cap V  = \emptyset,
%\end{align*}
%y, por la continuidad de $h$ se cumple que $h(U) \subset V$, de le definición de $\Kst$ estructura, existe $K$ tal que $K \subset U$ y por tanto $h(K) \subset V$ tenemos así que,
%
%\begin{align*}
%    K \cap h(K) = \emptyset.
%\end{align*}
%
%\end{proof}
%
%\begin{ob}
%Del lema anterior notemos que aplicando la función $h^{-1}$, la inversa de la función $h$, tenemos que
%
%\begin{align*}
%    K \cap h^{-1}(K) = \emptyset,
%\end{align*}
%
%teniendo así  que 
%
%\begin{align*}
%    K \cap (h^{-1}(K) \cup h(K)) = \emptyset.
%\end{align*}
%
%\end{ob}
%
%\begin{proof}
%Bajo las hipótesis del lema vamos a demostrar la conclusión del teorema, sea $g_0 \in Hom_0(X)$ con $K$ un conjunto de soporte para $g_0$ y $K$ un elemento de la $\Kst$ estructura, como $X$ tiene $(G,\Kst)$  estructura de la definción existe $\alpha \in G$ tal que 
%
%\begin{align*}
%    \alpha(K) = K_0.
%\end{align*}
%
%Tomando a $g_0= g^{[\alpha]}$ se tiene que 
%\end{proof}
